
\section{Introduction}

As byte-addressable non-volatile memories (BNVMs) become
common~\cite{lee_architecting_2009,fox:2001feram,intel3dxpoint}, it is
increasingly important that systems are optimized to leverage their
strengths and avoid stressing their weaknesses.  Historically, such
optimizations have included reducing the number of writes performed, either by
designing data structures that require fewer writes or by using hardware
techniques such as caching to reduce writes. However, it is the number of \emph{bits flipped} that
matter most for BNVMs such as phase-change memory (PCM), \emph{not} the number of words written.

%However, for BNVMs such as
%phase-change memory (PCM), it is the number of \emph{bits flipped} that matters
%most, not the number of words written.

BNVMs such as PCM suffer from two problems caused by flipping bits: energy usage
and cell wear-out.
As these memory technologies are adopted into longer-term storage solutions and
battery powered mobile and IoT devices, their costs become dominated by physical
replacement from wear-out and energy use respectively, so increasing lifetime
and dropping power consumption are vital optimizations for BNVM.
Flipping a bit in a PCM consumes
$15.7\mathit{-}22.5\times$ more power than reading a bit or ``writing''
a bit that does not actually
change~\cite{dhiman_pdram:_2009,lee_architecting_2009,xiangyu_dong_nvsim:_2012,qureshi_scalable_2009}.
Thus, many controllers optimize by only flipping bits when the value being
written to a cell differs from the old value~\cite{yang:iscas07}.  While this
approach saves some energy, it cannot eliminate flips required by software to
update modified data structures. An equally important concern is that PCM has
limited endurance: cells can only be written a limited number of times before
they ``wear out''. Unlike flash, however, PCM cells are written individually, so
it is possible (and even likely) that some cells will be written more than
others during a given period because of imbalances in values written by
software. Reducing bit flips, an optimization goal that has yet to be
sufficiently explored, can thus both save energy and extend the life of BNVM.

%There are optimization
%opportunities to reduce both wear and energy use at many levels of the stack;
%however, the exact optimization target from the perspective of software is no
%longer writes themselves, as was the case in previous technologies, but in
%the contents of the writes.
%
%These memory technologies, with their lower power consumption compared to DRAM,
%their potential for long-term but low latency persistent data storage, and their
%densities 
%yet existing approaches rarely attempt to
%optimize for flipping as few bits as possible, instead opting to reduce words
%written. Because it
%is bit flips that are expensive, however, data structures designed for BNVM must
%take into account bits flipped by their updates to properly evaluate
%how they will affect BNVM.


%Prior research~\cite{bittman:nvmsa18} has shown that small changes in data
Previously, we showed that small changes in data
structures can have large impacts in the bit flips required to
complete a given set of data structure modifications~\cite{bittman:nvmsa18}.
While it is possible to
reduce bits flipped with changes to hardware, we can gain more
by optimizing compiler constructs and choosing data structures to take advantage
of semantic information that is not available at other layers of the stack;
it is critical we design our data structures with this in mind. Successful BNVM-optimized
systems will need to target new optimizations for BNVM, including bit flip reduction.

We implemented three such
data structures and evaluated the impact on the number of writes and bit flips,
demonstrating the effectiveness of designing data structures to minimize bit
flips.
%These simple changes can extend the lifetime of BNVMs such as
%PCM by as much as $3.56\times$ and reduce power consumption by a similar amount,
These simple changes reduce bit flips by as much as $3.56\times$, and therefore will reduce power
consumption and extend lifetime by a proportional amount,
%can extend the lifetime of BNVMs such as
%PCM by as much as $3.56\times$ and reduce power consumption by a similar amount,
with no need to modify the hardware in any way.
Our contributions are:
\begin{enumerate}
	\item Implementation of bit flip counting in a full cycle-accurate simulation
	      environment to study bit
	      flip behavior.%, that we will make publicly available.
	\item Empirical evidence that reducing memory writes may not reduce bit flips
	      proportionally.
	\item Measurements of the number of bit flips required by operations such as
	      memory allocation and stack frame use, and suggestions for reducing the
	      bit flips they require.
	\item Modification of three data structures (linked lists, hash tables, red-black
	      trees) to reduce bit flips and evaluation of the effectiveness of the techniques.
\end{enumerate}

The paper is organized as follows. Section~\ref{sec:background} gives
background demonstrating how bit flips impact power consumption and
BNVM lifetime. Section~\ref{sec:datastruct} discusses some techniques for
reducing bit flips in software, which are evaluated for bit flips
(Section~\ref{sec:evaluation}) and performance (Section~\ref{sec:perf}).
Section~\ref{sec:disc} discusses the results, followed by comments on
future work (Section~\ref{sec:fw}) and a conclusion
(Section~\ref{sec:conclusion}).

