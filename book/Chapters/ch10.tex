
\chapter{Conclusion}\label{ch:conclusion}


\section{To Look Ahead}

\squo{``Where did you go to, if I may ask?'' said Thorin to Gandalf as they rode along.
    ``To look ahead,'' said he.
}{\emph{The Hobbit}, J.R.R Tolkien}

A common saying asserts that a dissertation describes completed work, and thus any discussion of future work should reflect
the nature that the work is completed. Yet I would be remiss if I didn't adequately convey that the nature of an
operating system does not include completeness.




\unedit{
    \paragraph{Compiler and Hardware Support.}
    \Twizzler's clean-slate \NVM abstraction reopens the possibility of coevolving OSes, compilers and
    languages, and hardware.
    Standardized OS support for cross-object pointers provides a stationary target for both
    compilers and hardware to design towards, whereas application-specific solutions do not.
    \Twizzler's pointer translation functions are simple enough to be emitted by a compiler. We plan
    to explore adding basic compiler support for C and C++ to automatically interoperate with
    \Twizzler so that persistent pointers are even more transparent to the compiler. Better still,
    we would like to study additional language-level support for persistent pointers, including type
    and lifetime annotations (such as the ones supported in Rust~\cite{rust}) for additional semantics the
    compiler can make use of when emitting code that operates directly on persistent data
    structures.

    Hardware support, too, can be helpful in improving the performance of our pointer translations.
    With \Twizzler providing a common framework, we can clearly state our needs to hardware. For
    example, hardware accelerated FOT access would improve the performance in pointer-heavy data
    structures. Segmentation support, allowing us to assign page-tables for each object and load
    them in as needed, would dramatically speedup memory mapping (and move memory management closer
    to the semantics of our programming model). Finally, first-class support for abstracting
    physical memory---a necessary feature for efficiently moving data around in a heterogeneous
    memory hierarchy in the face of numerous devices---would simplify the design of the kernel
    because we would not need to invoke the entirety of the virtualization hardware. We are
    interested in exploring modifications to RISC-V to better support \Twizzler.


    \paragraph{Security.} Although we discussed the \Twizzler security model briefly,
    there is still much to do. The current model provides access control, a basic ability to
    define and assign roles based on security contexts, and simple sub-process fault isolation through
    the ability to switch security contexts. We are exploring a \emph{flexible} security model
    that allows programmers to easily trade-off between security, transparency, and performance using
    capability-based verification. For
    example, we are implementing a call-gating mechanism that will allow us to
    restrict control-flow transfers between application components, improving the security against
    malicious components and reducing the possibility of memory-corrupting bugs.



    \paragraph{Networking and Distributed \Twizzler.}

    One of the key principles of \Twizzler is to focus the programming model on data and away from
    ephemeral actors such as processes and nodes. This is enabled by our identity-based pointers that
    decouple location from references,
    and by ensuring all the context
    necessary to understand these relationships is stored with the data. Because our data relationships
    are independent of the context of a particular machine, applications can more easily share data.
    This easy sharing, combined with a large ID address space, motivates a \emph{truly} global object ID
    space.

    We are building a networking stack and support for a distributed object space into \Twizzler. Our
    networking stack is based around extensive use of hardware virtualization in modern NICs. This
    design, which is in use in existing kernel-bypass strategies, will works well with our core OS design
    of reducing kernel interposition. At a higher level, we are
    considering how distributed applications change in our model. For example, an increase in data
    mobility facilitated by our location-independent data references and identities
    means that we can manifest both data and code where they are needed without complex marshalling, turning
    distributed computation into a rendezvous problem. We plan to build distributed applications atop
    \Twizzler to demonstrate this approach.

    Of course, for compatibility we will provide a traditional sockets-based networking stack. However,
    we can use existing userspace libraries that, \eg, implement TCP in userspace.
    Because we implemented our POSIX compatibility library in userspace, applications can gain many
    benefits afforded by kernel-bypass networking frameworks while still using traditional socket
    interfaces.


    \paragraph{Alternative Block Storage Technologies.}

    Our work meshes well with key-value SSDs, which extend the NVMe
    specification to include \texttt{put} and \texttt{get} operations. This would allow us to store and
    retrieve parts of objects based on their names rather than block addresses, thus greatly
    simplifying the storage system of the OS because it removes the need for a filesystem. \Twizzler
    uses a userspace pager design for moving data between memory and indirectly-accessible storage;
    providing a more ``native'' interface for object-based storage will greatly improve the performance
    of this system. We have prototype KVSSDs, and are in the process of implementing support for them in
    \Twizzler.

}



\section{Looking Behind}

\squo{``And what brought you back in the nick of time?''
    ``Looking behind,'' said he.
}{\emph{The Hobbit}, J.R.R Tolkien}






\section{Final Remarks}

\squo{One by one their seats were emptied,\\
    And one by one they went away;\\
    Now the family is parted,\\
    Will it be complete one day?\\
}{\emph{Will the Circle be Unbroken}, Ada R. Habershon}
