\pdfbookmark[1]{Foreword}{foreword}

\begingroup
\let\clearpage\relax
\let\cleardoublepage\relax
\let\cleardoublepage\relax
\chapter*{Foreword}

%\squo{You don’t learn about the important things in life from fabricated stories.}{Disco Elysium}

\squo{BASHIR: Out of all the stories you told me---which ones were true, and which ones weren't?\\
    GARAK: My dear doctor! They're \emph{all} true.\\
    BASHIR: Even the lies?\\
    GARAK: \emph{Especially} the lies.
}{Star Trek: Deep Space Nine}

This dissertation presents a narrative. That narrative is, at time of writing, the best understanding of the work that I
have. Any researcher knows that one doesn't start a research expedition with a fully formed understanding of a narrative
in-mind; instead, that narrative grows and changes dramatically with each success, but more importantly with each
failure. And while I'd like the narrative---the \emph{story}---of the work herein to contain not just the sapling that is the
product of my time at UCSC, but also the dead branches that I've pruned along the way, such story telling would distract
from the scientific communication of this work. But I'd like to expose a sample of that more accurate but no less ``true''
story here.

\squo{Audiences know what to expect, and that is all that they are prepared to believe in.}
{\emph{Rosencrantz and Guildenstern Are Dead}, Tom Stoppard}

One might well question the logic behind prefacing this disseratation with a refutation of its presented narrative.
However, I think it's important to consider what effect a work may have upon a reader. In particular, if even one
student reads these chapters and comes away with a belief that the scientific process (and, more specifically, the Ph.D.
process) is linear and \emph{not} an experience of one feeling their way through a dark labyrinth, then I will have been
negligent in my duty to help future students and avoid harming them.

\squo{All this happened, more or less.}{\emph{Slaughterhouse-Five}, Kurt Vonnegut}

Twizzler started as a combination of two things: my love for hacking kernels, and an idea for a new OS focused on the
exciting new technology of byte-addressable non-volatile memory (NVM). It has since grown well beyond that limited
scope, but at the time, that limitation allowed me to get started. We hadn't yet figured out the ``big picture'' words
to use to describe what we were trying to do, instead we presented the work as a straight-forward design to improve
performance for applications on NVM. It seemed like the perfect time, with Intel releasing 3D-Xpoint memory, and the
interest in NVM exploding. All I had to do was build an operating system to manifest the designs we'd been whiteboarding.

I started by prototyping the ideas inside FreeBSD, modifying the kernel to act like we wanted it to. This
was largely a dead end, since we rapidly ran into walls trying to force a \unix kernel to be something it wasn't. The
FreeBSD prototype was superceded with a custom kernel and grew into the main Twizzler operating system that is presented herein. It
needed to be reworked and redesinged several times, each time leading to me pulling long nights programming and writing.
But, in the end, the system worked, despite us not yet having the right words to describe it.

Since I mentioned failure earlier, let's talk about publications. It took us \emph{years} to get Twizzler published in a
conference. Along the way we had two workshop publications, but the rejection notifications from conferences were, shall
we say, starting to get to me. Fortunately, I had supportive professors who showed me their ``co-CVs'', containing lists of
rejected publications. I started keeping one as well. This is all to say---if you're a student, and you are struggling
to publish: we have all been there\sidenote{Anyone who claims to know, in full, what they are doing is a \emph{liar}.}, and (despite how some program committees behave) we are rooting for you.
Maybe one day our academic publishing system will function well.

I've also had blessings in disguise. As I said earlier, the initial writings on Twizzler framed the work nearly
exclusively around NVM and performance (``getting the kernel out of the way'').
Looming on the horizon, however, was the growing certainty that NVM (or,
at least, 3D-Xpoint) would \emph{not} soon deliver on its promises. Looking back, we got lucky. We were forced to
rethink the narrative, to look at the forest instead of the trees, and I think the work is much stronger as a result. We
were able to generalize. In fact, one of the benefits to the design presented in Chapter~\ref{ch:invariant} is that,
while we built it for NVM at the start, it is actually quite generalizable to larger contexts, and so much of the
technical design work for Twizzler wasn't lost during this rethinking of the narrative.

The research process is labyrinthian, and never ending.
The snapshot of work presented here is distilled into a story that is, in my opinion, straight-forward and compelling. But the
road to get here was long, filled with pot holes, dead ends, and complete reimaginings. I hope you enjoy reading it as
much as I have enjoyed making this crazy idea a reality.

\newcommand{\heart}{\ensuremath\heartsuit}
\begin{center}
    \heart \,
    \heart \,
    \heart \,
\end{center}

{
\hfill \emph{Daniel Bittman}
}
\endgroup

