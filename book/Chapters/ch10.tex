
\chapter{Conclusion}\label{ch:conclusion}

\squo{RIEBECK: I learned a lot, by the end of everything. The past is past, now, but that’s...you know, that’s okay! It’s never really gone completely. The future is always built on the past, even if we won’t get to see it.}{\emph{Outer Wilds}}

\emph{Let's talk about this operating system we just designed and built.}

\section{To Look Ahead}

\squo{``Where did you go to, if I may ask?'' said Thorin to Gandalf as they rode along.
    ``To look ahead,'' said he.
}{\emph{The Hobbit}, J.R.R Tolkien}


We stand before a crossroads. Before us is a convergence of trends, a great upset within the memory hierarchy, faster
and smarter networks and interconnects, the disaggregation of compute, and the
ever increasing demands for concurrency and performance from software. The friction to reexamine our models and evolve
our operating systems is low---provided we can produce a compelling reason for developers to invest in new ways of programming.
Do we take the easy route, like so many times before, and try to shoehorn new technologies into existing interfaces, or
do we take a step towards a model that expresses a data-centric view and better fits with hardware trends and the goals
of software?

\vspace{5mm}
\noindent For us, the answer is clear\sidenote{It's the second thing.}.
\vspace{5mm}

So, what happens next? There are two aspects of the future I want to discuss with respect to Twizzler---the future of
the model we discussed and its resilience, and future research directions for Twizzler.

\subsection{Future Models}

Looking back, it really does appear, in hindsight, as if operating system interfaces plunged fully and purposefully into the endless maw of
complexity. However, the decisions over the years were not made with the foresight required to predict what criticisms
we today would aim at those choices. The model I have presented here is similarly a product of its time, made with
some attempt at foresight but, ultimately, must be superceded when it, too, becomes too unwieldy to effectively use
among new hardware and software characteristics. That said, I have made some attempts to design a general model that
may be useful beyond \emph{specific} hardware and instead is modeled after patterns:

\begin{enumerate}
    \item Instead of focusing specifically on NVM DIMMs, we have designed our object model around the idea of a
          heterogeneous memory system in general. Should future kinds of memory be adopted, and mix-and-matched, or should
          NUMA become more common, our model built for understanding data movement within physical memory as a first-class
          operation will be able to adapt.
    \item Instead of trying to be doctrinaire at a low-level and limit our invariant references to the semantics of a
          single language, we provide a ``lowest common denominator'' in our implementation of references, and allow
          additional metadata to be stored in the FOT should richer semantics need additional storage. As a result, we can not
          only support languages of today, but also future languages and runtimes that may wish for a richer model.
    \item The Twizzler kernel is small, largely because most functionality is pushed into userspace. Thus, not only can
          we more easily modularize, update, and replace components, but those components are closer to applications and can
          be communicated with more easily.
    \item The design of our global address space is not limited to exist within a single machine, it is large enough
          that huge numbers of computers can interact with it wih low coordination. Should the size need to increase, we can
          do so with an application-agnostic procedure. Finally, the address space is usable both by multiple
          independent nodes, but also components within the nodes, enabling hardware devices to operate independently.
\end{enumerate}

\subsection{Future Research}

\squo{I hope that we continue with exploration.}{Margaret H. Hamilton}

A common saying asserts that a dissertation describes completed work, and thus any discussion of future work should reflect
the nature that the work is completed. Yet I would be remiss if I didn't adequately convey that the nature of an
operating system does not include completeness, and for a research operating system, that means there are always future
research directions we can explore. Here are a few.

\paragraph{Compiler Support}
\Twizzler's clean-slate \NVM abstraction reopens the possibility of co-evolving OSes, compilers and
languages, and hardware.
Standardized OS support for cross-object pointers provides a stationary target for both
compilers and hardware to design towards, whereas application-specific solutions do not.
\Twizzler's pointer translation functions are simple enough to be emitted by a compiler. We plan
to explore adding basic compiler support for C and C++ to automatically interoperate with
\Twizzler so that persistent pointers are even more transparent to the compiler. Better still,
we would like to study additional language-level support for persistent pointers, including type
and lifetime annotations, such as the ones supported in Rust~\cite{rust}, for additional semantics the
compiler can make use of when emitting code that operates directly on persistent data
structures.

\paragraph{Hardware Support}
Hardware support, too, can be helpful in improving the performance of our pointer translations.
With \Twizzler providing a common framework, we can clearly state our needs to hardware. For
example, hardware accelerated FOT access would improve the performance of pointer-heavy data
structures. Segmentation support, allowing us to assign page-tables for each object and load
them in as needed, would dramatically speedup memory mapping (and move memory management closer
to the semantics of our programming model). Finally, first-class support for abstracting
physical memory---a necessary feature for efficiently moving data around in a heterogeneous
memory hierarchy in the face of numerous devices---would simplify the design of the kernel
because we would not need to invoke the entirety of the virtualization hardware. We are
interested in exploring modifications to RISC-V to better support \Twizzler.

\paragraph{Higher Level Programming Frameworks}

The programming model we discussed in Chapter~\ref{ch:prog} provides a number of basic operations on objects, but one
could imagine a higher level model. Implementing Rust APIs for more complex transactional semantics on shared objects
would allow programmers to easily express atomic operations. A higher level framework that has awareness of application
semantics via the FOT can perform optimizations on behalf of the application that would traditionally be difficult to
implement, \eg semantics-aware prefetching and caching in cooperation with the network, enabled by inspecting the FOT. A
framework that also has knowledge of operations that an application may perform on an object could then perform
different distribution and coherence strategies for CRDT~\cite{shapiro2011comprehensive} objects than for immutable or
arbitrarily-mutable objects.

\iffalse

    \paragraph{Security.} Although we discussed the \Twizzler security model briefly,
    there is still much to do. The current model provides access control, a basic ability to
    define and assign roles based on security contexts, and simple sub-process fault isolation through
    the ability to switch security contexts. We are exploring a \emph{flexible} security model
    that allows programmers to easily trade-off between security, transparency, and performance using
    capability-based verification. For
    example, we are implementing a call-gating mechanism that will allow us to
    restrict control-flow transfers between application components, improving the security against
    malicious components and reducing the possibility of memory-corrupting bugs.



    \paragraph{Networking and Distributed \Twizzler.}

    One of the key principles of \Twizzler is to focus the programming model on data and away from
    ephemeral actors such as processes and nodes. This is enabled by our identity-based pointers that
    decouple location from references,
    and by ensuring all the context
    necessary to understand these relationships is stored with the data. Because our data relationships
    are independent of the context of a particular machine, applications can more easily share data.
    This easy sharing, combined with a large ID address space, motivates a \emph{truly} global object ID
    space.

    We are building a networking stack and support for a distributed object space into \Twizzler. Our
    networking stack is based around extensive use of hardware virtualization in modern NICs. This
    design, which is in use in existing kernel-bypass strategies, will works well with our core OS design
    of reducing kernel interposition. At a higher level, we are
    considering how distributed applications change in our model. For example, an increase in data
    mobility facilitated by our location-independent data references and identities
    means that we can manifest both data and code where they are needed without complex marshalling, turning
    distributed computation into a rendezvous problem. We plan to build distributed applications atop
    \Twizzler to demonstrate this approach.

    Of course, for compatibility we will provide a traditional sockets-based networking stack. However,
    we can use existing userspace libraries that, \eg, implement TCP in userspace.
    Because we implemented our POSIX compatibility library in userspace, applications can gain many
    benefits afforded by kernel-bypass networking frameworks while still using traditional socket
    interfaces.

\fi

\paragraph{Alternative Storage Technologies}

Twizzler meshes well with key-value SSDs, which extend the NVMe
specification to include \texttt{put} and \texttt{get} operations. This would allow us to store and
retrieve parts of objects based on their names rather than block addresses, thus greatly
simplifying the storage system of the OS because it removes the need for a filesystem. \Twizzler
uses a userspace pager design for moving data between memory and indirectly-accessible storage;
providing a more ``native'' interface for object-based storage will greatly improve the performance
of this system.



\section{Looking Behind}

\squo{``And what brought you back in the nick of time?''
    ``Looking behind,'' said he.
}{\emph{The Hobbit}, J.R.R Tolkien}

Let's return to the principal hypothesis I put forward in Chapter~\ref{ch:intro}.

\begin{quotation}
    \noindent The data-centric model
    for designing operating system abstractions is not only viable, but demanded both by software and hardware trends.
\end{quotation}

In Chapters~\ref{ch:farout} and~\ref{ch:softwaredemands} we discussed the trends in hardware and software that led to
the above statement, putting together the arguments that, (1) hardware is pushing us towards having in-memory
data structures escape their traditional confinement, (2) software overheads induced by the context problem and
operating system involvement are too high for modern hardware, (3) a data-centric model addresses the problems we have
discussed via a method of progamming and storing data that requires no serialization and enables direct access and
sharing. We also discussed why retrofitting is insufficient, covering Claim 1 from Chapter~\ref{ch:intro}.

\squo{“The most important reason for going from one place to another is to see what's in between.”}{\emph{The Phantom
        Tollbooth}, Norton Juster}
In Chapter~\ref{ch:datacentric} we discussed some details of the data-centric viewpoint, followed by elucidating
Twizzler's place in the design space. Chapters~\ref{ch:global} and~\ref{ch:invariant} covered the global address
space and invariant references, going into detail with case studies, performance analysis, and modeling to
establish the viability of the model in terms of usability, and performance and space overhead, thus covering the
remaining Chapter~\ref{ch:intro} Claims.

Chapters~\ref{ch:twizzler} and~\ref{ch:prog} wrap up by discussing operating system services and implementation, and
programming model details. Discussing details like failure-atomicity in allocation, persistence, durability, and
crash-recovery is vital, as it establishes where the complexity in the system still remains, despite many things getting
easier.


\section{Final Remarks}

\squo{SOLANUM: It’s tempting to linger in this moment, while every possibility still exists. But unless they are
    collapsed by an observer, they will never be more than possibilities.}{\emph{Outer Wilds}}
%\squo{One by one their seats were emptied,\\
%    And one by one they went away;\\
%   Now the family is parted,\\
%    Will it be complete one day?\\
%}{\emph{Will the Circle be Unbroken}, Ada R. Habershon}


Operating systems must evolve to support future trends in memory hierarchy
organization. Failing to evolve will relegate new technology to
outdated access models, preventing it from reaching its full potential,
and making it difficult for OSes to evolve in the
future. \Twizzler shows a way forward: an
OS designed around new hardware trends and software demands that provides
new, efficient, and easy to use semantics for direct access to memory in a global address space.
\Twizzler will give us a system from which
we can build a full \NVM-based OS around a data-centric design and explore the future of applications, OSes, and
processor design on a new memory hierarchy.

\squo{...How beautiful. It’s different than I’d envisioned. Whatever happens next, I do not think it is to be feared.}{The Prisoner, \emph{Outer Wilds}}

Overall, we have shown that invariant pointers in \Twizzler allow programmers to easily build
composable and extensible applications with low overhead
by removing the kernel from persistent data access paths,
thereby improving the flexibility and performance.
Our simpler programming model improved performance despite the small
pointer translation overhead.
Twizzler is easy to work with
compared to existing systems, and we were able to both quickly prototype real
applications with advanced access control features and port existing software, such as SQLite.

But those aren't the real lessons. Instead, consider how much of the operating system was under review after asking a
simple question, ``what if the lifetime of in-memory data extends beyond the ephemera?'' This question was at the heart
of this research, and led to all the work herein. It's not tied to a specific idea about persistent
memory\sidenote{Though I won't deny that NVM kicked it off.} or networking; instead, it's about programming, applications, and
\emph{data}. Reconsidering core ideas in programming and operating systems can lead to dramatic shifts in our understanding and
designs. If we do not take the plunge by calling into question core beliefs and make moves towards building new things,
we will be forever mired in the cage of our outdated systems. But there is hope---not from Twizzler alone, but from all
the operating systems research that has exploded onto the scene.

\vspace{5mm}
\squo{``It's a magical world, Hobbes, ol' buddy... Let's go exploring!''}{Calvin, \emph{Calvin and Hobbes},\\Bill Watterson}

\noindent It's an exciting time, and I cannot wait for the next
new operating system that asks, \emph{what if things could be better}.

