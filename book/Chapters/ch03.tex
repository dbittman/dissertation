
\chapter{The Demands of Software}\label{ch:softwaredemands}

\section{The Old Ways}



\unedit{
    %will need to fix the R1 and R2 here, and this might involve moving around stuff in this block
    Current OS techniques do not meet requirements R1 and R2 as we set out above---file
    \texttt{read} and \texttt{write} interfaces,
    designed for sequential media and later expanded for block-based media,
    require significant kernel involvement and serialization, violating both requirements.
    %, and serialization that requires
    %the programmer to add complexity and overhead through multiple data representations and transformations.
    While support for these interfaces can be useful for legacy applications, as we will demonstrate,
    providing the programmer with abstractions designed \emph{for} \NVM both reduces complexity and
    improves performance.

    The \texttt{mmap} system call attempts to hide storage behind a memory interface through hidden
    data copies. But, with \NVM, these copies are wasteful, and \texttt{mmap} still has significant kernel
    involvement and the need for explicit \texttt{msync} calls. ``Direct Access'' (DAX)
    tries to retrofit \texttt{mmap} for \NVM by removing the redundant copy, but this \emph{still}
    fails to address requirement R2! Operating on persistent data through \texttt{mmap} requires the programmer to
    use either fixed virtual addresses, which presents an infeasible coordination problem
    as we scale across machines, or virtual addresses directly, which are ephemeral and require the
    context of the process that created them.



    Attempting to shoehorn \NVM programming atop POSIX interfaces (including \texttt{mmap}) results in
    complexity that arises from combining multiple partial solutions. Given some feature desired by an
    application, the \NVM framework can provide an integrated solution that meshes
    well with the existing support for persistent data structure manipulation and access, or it can
    fall-back to POSIX resulting in the programmer needing to understand two different ``feature
    namespaces'' and their interactions. An example of this is naming, where a programmer may need
    to turn to the filesystem to manage names in a completely orthogonal way to how the \NVM frameworks handles
    data references. For example, PMDK, a \NVM programming library, relies on a filesystem for naming and initial access to
    persistent memory objects, resulting in different kinds of references, feature sets from filesystems
    being applied (like security) while others are not (data access), and the complexity of
    understanding how the PMDK abstractions interact with the POSIX ones. Instead, our model prefers to
    build legacy support atop new abstractions (\S~\ref{sec:fbsdimpl}), and avoid falling back to legacy
    models for persistent data access. We will discuss another example,
    security, in our case study (\S~\ref{sec:eval}).


    Additionally, models that layer \NVM programming atop existing interfaces often fail to facilitate effective persistent data sharing and
    protection.  PMDK, for example, makes design choices that limit
    scalability, since its
    data objects are not self-contained and do not have a large enough ID space, resulting
    in the need to coordinate object IDs across machines~\cite{bittman:plos19}. For the same reason,
    although single-address space OSes~\cite{chase:tocs94} somewhat address requirement R1, they do
    not consider both requirements at once, nor do they provide an effective and scalable solution to
    long-term data references due to that same coordination complexity~\cite{bittman:hotstorage19}.


}




\section{Serialization}

\section{Remote Procedure Calls}