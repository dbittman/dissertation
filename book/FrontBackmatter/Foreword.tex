\pdfbookmark[1]{Foreword}{foreword}

\begingroup
\let\clearpage\relax
\let\cleardoublepage\relax
\let\cleardoublepage\relax
\chapter*{Foreword}

\todo[inline]{actually write this section lol}

\squo{BASHIR: Out of all the stories you told me---which ones were true, and which ones weren't?\\
    GARAK: My dear doctor! They're \emph{all} true.\\
    BASHIR: Even the lies?\\
    GARAK: \emph{Especially} the lies.
}{Star Trek: Deep Space Nine}

\squo{All this happened, more or less.}{\emph{Slaughterhouse-Five}, Kurt Vonnegut}

\squo{Audiences know what to expect, and that is all that they are prepared to believe in.}
{\emph{Rosencrantz and Guildenstern Are Dead}, Tom Stoppard}

%\squo{You don’t learn about the important things in life from fabricated stories.}{Disco Elysium}

\paragraph{The Honest Narrative}todo


One might well question the logic behind prefacing this disseratation with a refutation of its presented narrative.
However, I think it's important to consider what effect a work may have upon a reader. In particular, if even one
student reads these chapters and comes away with a belief that the scientific process (and, more specifically, the Ph.D.
process) is linear and \emph{not} an experience of one feeling their way through a dark labyrinth, then I will have been
negligent in my duty to help future students and avoid harming them.

\paragraph{The Pandemic}

todo
\squo{``I wish it need not have happened in my time,'' said Frodo.
    ``So do I,'' said Gandalf, ``and so do all who live to see such times. But that is not for them to decide. All we have
    to decide is what to do with the time that is given us.''}{\emph{The Lord of the Rings}, J.R.R. Tolkien}


\endgroup

