
\section{Future Work}
\label{sec:fw}

Although we covered a range of different data structures, there are many more
used in storage systems that should be examined, such as
B-trees~\cite{btree} and
LSM-trees~\cite{lsmtree}, both to understand their bit flipping behavior as compared to
other data structures and to examine for potential optimizations. In
addition to data structures, different algorithms such as sorting can be
evaluated for bit flips. Though this may come
down to data movement minimization, there may be optimizations in locality that
could affect bit flips.

While data structure and algorithm evaluation can provide system designers with
insights for how to reduce bit flips, examining bit flips in a large system,
including one that properly implements consistency and our suggested stack frame modifications
(perhaps through compiler modification),
would be worthwhile.
There are a number of \NVM-based key-value stores~\cite{kv1}; comparing them
for bit flips could demonstrate the benefits of some designs
over others.

Studying bit flips directly is a good metric for understanding
power consumption and wear, but a better understanding through the evaluation of
real \NVM would be illuminating. The power study discussed earlier was
derived from a number of research papers that give approximate numbers or
estimates. On a real system, we could \textit{measure} power
consumption, and cooperation with vendors may enable accurate studies of
wear caused by bit flips. Additionally, some technologies (\eg, PCM) have a disparity between writing
a 1 or a 0, something that could be leveraged by software (in cooperation with hardware) to
further optimize power use.

