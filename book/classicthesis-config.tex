% ****************************************************************************************************
% classicthesis-config.tex
% formerly known as loadpackages.sty, classicthesis-ldpkg.sty, and classicthesis-preamble.sty
% Use it at the beginning of your ClassicThesis.tex, or as a LaTeX Preamble
% in your ClassicThesis.{tex,lyx} with % ****************************************************************************************************
% classicthesis-config.tex
% formerly known as loadpackages.sty, classicthesis-ldpkg.sty, and classicthesis-preamble.sty
% Use it at the beginning of your ClassicThesis.tex, or as a LaTeX Preamble
% in your ClassicThesis.{tex,lyx} with % ****************************************************************************************************
% classicthesis-config.tex
% formerly known as loadpackages.sty, classicthesis-ldpkg.sty, and classicthesis-preamble.sty
% Use it at the beginning of your ClassicThesis.tex, or as a LaTeX Preamble
% in your ClassicThesis.{tex,lyx} with % ****************************************************************************************************
% classicthesis-config.tex
% formerly known as loadpackages.sty, classicthesis-ldpkg.sty, and classicthesis-preamble.sty
% Use it at the beginning of your ClassicThesis.tex, or as a LaTeX Preamble
% in your ClassicThesis.{tex,lyx} with \input{classicthesis-config}
% ****************************************************************************************************
% If you like the classicthesis, then I would appreciate a postcard.
% My address can be found in the file ClassicThesis.pdf. A collection
% of the postcards I received so far is available online at
% http://postcards.miede.de
% ****************************************************************************************************


% ****************************************************************************************************
% 0. Set the encoding of your files. UTF-8 is the only sensible encoding nowadays. If you can't read
% äöüßáéçèê∂åëæƒÏ€ then change the encoding setting in your editor, not the line below. If your editor
% does not support utf8 use another editor!
% ****************************************************************************************************
\PassOptionsToPackage{utf8}{inputenc}
\usepackage{inputenc}

\PassOptionsToPackage{T1}{fontenc} % T2A for cyrillics
\usepackage{fontenc}


% ****************************************************************************************************
% 1. Configure classicthesis for your needs here, e.g., remove "drafting" below
% in order to deactivate the time-stamp on the pages
% (see ClassicThesis.pdf for more information):
% ****************************************************************************************************
\PassOptionsToPackage{
  drafting=true,    % print version information on the bottom of the pages
  tocaligned=false, % the left column of the toc will be aligned (no indentation)
  dottedtoc=true,  % page numbers in ToC flushed right
  eulerchapternumbers=true, % use AMS Euler for chapter font (otherwise Palatino)
  linedheaders=false,       % chaper headers will have line above and beneath
  floatperchapter=true,     % numbering per chapter for all floats (i.e., Figure 1.1)
  eulermath=true,  % use awesome Euler fonts for mathematical formulae (only with pdfLaTeX)
  beramono=true,    % toggle a nice monospaced font (w/ bold)
  palatino=false,    % deactivate standard font for loading another one, see the last section at the end of this file for suggestions
  style=classicthesis % classicthesis, arsclassica
}{classicthesis}


% ****************************************************************************************************
% 2. Personal data and user ad-hoc commands (insert your own data here)
% ****************************************************************************************************
\newcommand{\myTitle}{Operating Systems for Far Out Memories\xspace}
\newcommand{\mySubtitle}{Building a new OS for Upcoming Hardware\xspace}
%\newcommand{\myDegree}{Doktor-Ingenieur (Dr.-Ing.)\xspace}
\newcommand{\myName}{Daniel Bittman\xspace}
%\newcommand{\myProf}{Put name here\xspace}
%\newcommand{\myOtherProf}{Put name here\xspace}
%\newcommand{\mySupervisor}{Put name here\xspace}
%\newcommand{\myFaculty}{Put data here\xspace}
\newcommand{\myDepartment}{Computer Science\xspace}
\newcommand{\myUni}{University of California, Santa Cruz\xspace}
\newcommand{\myLocation}{Santa Cruz, CA\xspace}
\newcommand{\myTime}{November 2022\xspace}
\newcommand{\myVersion}{1.0}

% ********************************************************************
% Setup, finetuning, and useful commands
% ********************************************************************
\providecommand{\mLyX}{L\kern-.1667em\lower.25em\hbox{Y}\kern-.125emX\@}
\newcommand{\ie}{\textit{i.\,e.}}
\newcommand{\Ie}{\textit{I.\,e.}}
\newcommand{\eg}{\textit{e.\,g.}}
\newcommand{\Eg}{\textit{E.\,g.}}
\newcommand{\etc}{\textit{etc.}}
% ****************************************************************************************************


% ****************************************************************************************************
% 3. Loading some handy packages
% ****************************************************************************************************
% ********************************************************************
% Packages with options that might require adjustments
% ********************************************************************
\PassOptionsToPackage{ngerman,american}{babel} % change this to your language(s), main language last
% Spanish languages need extra options in order to work with this template
%\PassOptionsToPackage{spanish,es-lcroman}{babel}
\usepackage{babel}

\usepackage{csquotes}
\PassOptionsToPackage{%
  backend=biber,bibencoding=utf8, %instead of bibtex
  %backend=bibtex8,bibencoding=ascii,%
  language=auto,%
  style=numeric-comp,%
  %style=authoryear-comp, % Author 1999, 2010
  %bibstyle=authoryear,dashed=false, % dashed: substitute rep. author with ---
  sorting=nyt, % name, year, title
  maxbibnames=10, % default: 3, et al.
  %backref=true,%
  natbib=true % natbib compatibility mode (\citep and \citet still work)
}{biblatex}
\usepackage{biblatex}

\PassOptionsToPackage{fleqn}{amsmath}       % math environments and more by the AMS
\usepackage{amsmath}

% ********************************************************************
% General useful packages
% ********************************************************************
\usepackage{graphicx} %
\usepackage{scrhack} % fix warnings when using KOMA with listings package
\usepackage{xspace} % to get the spacing after macros right
\PassOptionsToPackage{printonlyused,smaller}{acronym}
\usepackage{acronym} % nice macros for handling all acronyms in the thesis
%\renewcommand{\bflabel}[1]{{#1}\hfill} % fix the list of acronyms --> no longer working
%\renewcommand*{\acsfont}[1]{\textsc{#1}}
%\renewcommand*{\aclabelfont}[1]{\acsfont{#1}}
%\def\bflabel#1{{#1\hfill}}
\def\bflabel#1{{\acsfont{#1}\hfill}}
\def\aclabelfont#1{\acsfont{#1}}
% ****************************************************************************************************
%\usepackage{pgfplots} % External TikZ/PGF support (thanks to Andreas Nautsch)
%\usetikzlibrary{external}
%\tikzexternalize[mode=list and make, prefix=ext-tikz/]
% ****************************************************************************************************


% ****************************************************************************************************
% 4. Setup floats: tables, (sub)figures, and captions
% ****************************************************************************************************
\usepackage{tabularx} % better tables
\setlength{\extrarowheight}{3pt} % increase table row height
\newcommand{\tableheadline}[1]{\multicolumn{1}{l}{\spacedlowsmallcaps{#1}}}
\newcommand{\myfloatalign}{\centering} % to be used with each float for alignment
\usepackage{subfig}
% ****************************************************************************************************


% ****************************************************************************************************
% 5. Setup code listings
% ****************************************************************************************************
\usepackage{listings}
%\lstset{emph={trueIndex,root},emphstyle=\color{BlueViolet}}%\underbar} % for special keywords
\lstset{language=[LaTeX]Tex,%C++,
  morekeywords={PassOptionsToPackage,selectlanguage},
  keywordstyle=\color{RoyalBlue},%\bfseries,
  basicstyle=\small\ttfamily,
  %identifierstyle=\color{NavyBlue},
  commentstyle=\color{Green}\ttfamily,
  stringstyle=\rmfamily,
  numbers=none,%left,%
  numberstyle=\scriptsize,%\tiny
  stepnumber=5,
  numbersep=8pt,
  showstringspaces=false,
  breaklines=true,
  %frameround=ftff,
  %frame=single,
  belowcaptionskip=.75\baselineskip
  %frame=L
}
% ****************************************************************************************************


\PassOptionsToPackage{margincaption,outercaption,ragged,wide,centerbody}{sidecap}
\RequirePackage{sidecap}
\sidecaptionvpos{figure}{t}
\sidecaptionvpos{table}{t}

\DeclareCaptionLabelFormat{slsc}{\spacedlowsmallcaps{#1 #2}}
\DeclareCaptionLabelSeparator{spacednewline}{\\[1ex]}
\captionsetup[SCfigure]{format=plain,labelsep=spacednewline,labelfont={rm,small},labelformat=slsc,textfont=rm,font=footnotesize,singlelinecheck=true,position=top}
\captionsetup[SCtable]{format=plain,labelsep=spacednewline,labelfont={rm,small},labelformat=slsc,textfont=rm,font=footnotesize,singlelinecheck=true,position=top}

% ****************************************************************************************************
% 6. Last calls before the bar closes
% ****************************************************************************************************
% ********************************************************************
% Her Majesty herself
% ********************************************************************
\usepackage{classicthesis}


% ********************************************************************
% Fine-tune hyperreferences (hyperref should be called last)
% ********************************************************************
\hypersetup{%
  %draft, % hyperref's draft mode, for printing see below
  colorlinks=true, linktocpage=true, pdfstartpage=3, pdfstartview=FitV,%
  % uncomment the following line if you want to have black links (e.g., for printing)
  %colorlinks=false, linktocpage=false, pdfstartpage=3, pdfstartview=FitV, pdfborder={0 0 0},%
  breaklinks=true, pageanchor=true,%
  pdfpagemode=UseNone, %
  % pdfpagemode=UseOutlines,%
  plainpages=false, bookmarksnumbered, bookmarksopen=true, bookmarksopenlevel=1,%
  hypertexnames=true, pdfhighlight=/O,%nesting=true,%frenchlinks,%
  urlcolor=CTurl, linkcolor=CTlink, citecolor=CTcitation, %pagecolor=RoyalBlue,%
  %urlcolor=Black, linkcolor=Black, citecolor=Black, %pagecolor=Black,%
  pdftitle={\myTitle},%
  pdfauthor={\textcopyright\ \myName, \myUni},%
  pdfsubject={},%
  pdfkeywords={},%
  pdfcreator={pdfLaTeX},%
  pdfproducer={LaTeX with hyperref and classicthesis}%
}


% ********************************************************************
% Setup autoreferences (hyperref and babel)
% ********************************************************************
% There are some issues regarding autorefnames
% http://www.tex.ac.uk/cgi-bin/texfaq2html?label=latexwords
% you have to redefine the macros for the
% language you use, e.g., american, ngerman
% (as chosen when loading babel/AtBeginDocument)
% ********************************************************************
\makeatletter
\@ifpackageloaded{babel}%
{%
  \addto\extrasamerican{%
    \renewcommand*{\figureautorefname}{Figure}%
    \renewcommand*{\tableautorefname}{Table}%
    \renewcommand*{\partautorefname}{Part}%
    \renewcommand*{\chapterautorefname}{Chapter}%
    \renewcommand*{\sectionautorefname}{Section}%
    \renewcommand*{\subsectionautorefname}{Section}%
    \renewcommand*{\subsubsectionautorefname}{Section}%
  }%
  \addto\extrasngerman{%
    \renewcommand*{\paragraphautorefname}{Absatz}%
    \renewcommand*{\subparagraphautorefname}{Unterabsatz}%
    \renewcommand*{\footnoteautorefname}{Fu\"snote}%
    \renewcommand*{\FancyVerbLineautorefname}{Zeile}%
    \renewcommand*{\theoremautorefname}{Theorem}%
    \renewcommand*{\appendixautorefname}{Anhang}%
    \renewcommand*{\equationautorefname}{Gleichung}%
    \renewcommand*{\itemautorefname}{Punkt}%
  }%
  % Fix to getting autorefs for subfigures right (thanks to Belinda Vogt for changing the definition)
  \providecommand{\subfigureautorefname}{\figureautorefname}%
}{\relax}
\makeatother


% ********************************************************************
% Development Stuff
% ********************************************************************
\listfiles
%\PassOptionsToPackage{l2tabu,orthodox,abort}{nag}
%  \usepackage{nag}
%\PassOptionsToPackage{warning, all}{onlyamsmath}
%  \usepackage{onlyamsmath}


% ****************************************************************************************************
% 7. Further adjustments (experimental)
% ****************************************************************************************************
% ********************************************************************
% Changing the text area
% ********************************************************************

% Palatino  10pt: 288--312pt | 609--657pt
%\areaset[current]{312pt}{761pt} % 686 (factor 2.2) + 33 head + 42 head \the\footskip
\areaset[current]{312pt}{657pt} % 686 (factor 2.2) + 33 head + 42 head \the\footskip
%\setlength{\marginparwidth}{7em}%
%\setlength{\marginparsep}{2em}%

% ********************************************************************
% Using different fonts
% ********************************************************************
%\usepackage[oldstylenums]{kpfonts} % oldstyle notextcomp
\usepackage[osf,tt=false]{libertine}
%\usepackage[light,condensed,math]{iwona}
%\renewcommand{\sfdefault}{iwona}
%\usepackage{lmodern} % <-- no osf support :-(
%\usepackage{cfr-lm} %
%\usepackage[urw-garamond]{mathdesign} <-- no osf support :-(
%\usepackage[default,osfigures]{opensans} % scale=0.95
%\usepackage{FiraSans}
%\usepackage[opticals,mathlf]{MinionPro} % onlytext

% TODO use euler-digits (and in matplotlibrc)?
\usepackage[small]{eulervm}
% ********************************************************************
%\usepackage[largesc,osf]{newpxtext}
%\linespread{1.05} % a bit more for Palatino
% Used to fix these:
% https://bitbucket.org/amiede/classicthesis/issues/139/italics-in-pallatino-capitals-chapter
% https://bitbucket.org/amiede/classicthesis/issues/45/problema-testatine-su-classicthesis-style
% ********************************************************************
% ****************************************************************************************************




\usepackage[letterpaper, inner=1.5in, outer=3in, top=1.25in, bottom=1.25in]{geometry}

\setlength{\marginparwidth}{11em}%
\setlength{\marginparsep}{2.5em}%


\edef\sidecaptionsep{\the\marginparsep}
\usepackage{sidenotes}



\newcommand{\etal}{\emph{et~al.}\xspace}
\newcommand{\unix}{\textsc{Unix}\xspace}
\newcommand{\Twizzler}{Twizzler\xspace}
\newcommand{\NVM}{NVM\xspace}

\newcommand{\unixkv}{\texttt{unixkv}\xspace}
\newcommand{\nvkv}{\texttt{twzkv}\xspace}
\newcommand{\ramrbt}{\texttt{ramrbt}\xspace}
\newcommand{\unixrbt}{\texttt{unixrbt}\xspace}
\newcommand{\nvrbt}{\texttt{twzrbt}\xspace}

\newcommand{\dab}[1]{{\textcolor{cyan}{[[#1 -- dab]]}}}


\newcommand{\unedit}[1]{{\leavevmode\color{red}#1}}

%%%%%%
% TODO: get rid of these
\newcommand{\observe}[2]{{\leavevmode\color{red}#2}}
\newcommand{\observation}[1]{{\leavevmode\color{red}#1}}
\newcommand{\observations}[1]{{\leavevmode\color{red}#1}}
\makeatletter
\newcommand\footnoteref[1]{\protected@xdef\@thefnmark{\ref{#1}}\@footnotemark}
\makeatother
%%%%%%

\newcommand{\libcore}{\texttt{libtwz}\xspace}

\usepackage{siunitx}
\usepackage{amsmath}
\usepackage{rotating}
\usepackage[pdf]{graphviz}
\usepackage{todonotes}
\usepackage{annotate-equations}
% ****************************************************************************************************
% If you like the classicthesis, then I would appreciate a postcard.
% My address can be found in the file ClassicThesis.pdf. A collection
% of the postcards I received so far is available online at
% http://postcards.miede.de
% ****************************************************************************************************


% ****************************************************************************************************
% 0. Set the encoding of your files. UTF-8 is the only sensible encoding nowadays. If you can't read
% äöüßáéçèê∂åëæƒÏ€ then change the encoding setting in your editor, not the line below. If your editor
% does not support utf8 use another editor!
% ****************************************************************************************************
\PassOptionsToPackage{utf8}{inputenc}
\usepackage{inputenc}

\PassOptionsToPackage{T1}{fontenc} % T2A for cyrillics
\usepackage{fontenc}


% ****************************************************************************************************
% 1. Configure classicthesis for your needs here, e.g., remove "drafting" below
% in order to deactivate the time-stamp on the pages
% (see ClassicThesis.pdf for more information):
% ****************************************************************************************************
\PassOptionsToPackage{
  drafting=true,    % print version information on the bottom of the pages
  tocaligned=false, % the left column of the toc will be aligned (no indentation)
  dottedtoc=true,  % page numbers in ToC flushed right
  eulerchapternumbers=true, % use AMS Euler for chapter font (otherwise Palatino)
  linedheaders=false,       % chaper headers will have line above and beneath
  floatperchapter=true,     % numbering per chapter for all floats (i.e., Figure 1.1)
  eulermath=true,  % use awesome Euler fonts for mathematical formulae (only with pdfLaTeX)
  beramono=true,    % toggle a nice monospaced font (w/ bold)
  palatino=false,    % deactivate standard font for loading another one, see the last section at the end of this file for suggestions
  style=classicthesis % classicthesis, arsclassica
}{classicthesis}


% ****************************************************************************************************
% 2. Personal data and user ad-hoc commands (insert your own data here)
% ****************************************************************************************************
\newcommand{\myTitle}{Operating Systems for Far Out Memories\xspace}
\newcommand{\mySubtitle}{Building a new OS for Upcoming Hardware\xspace}
%\newcommand{\myDegree}{Doktor-Ingenieur (Dr.-Ing.)\xspace}
\newcommand{\myName}{Daniel Bittman\xspace}
%\newcommand{\myProf}{Put name here\xspace}
%\newcommand{\myOtherProf}{Put name here\xspace}
%\newcommand{\mySupervisor}{Put name here\xspace}
%\newcommand{\myFaculty}{Put data here\xspace}
\newcommand{\myDepartment}{Computer Science\xspace}
\newcommand{\myUni}{University of California, Santa Cruz\xspace}
\newcommand{\myLocation}{Santa Cruz, CA\xspace}
\newcommand{\myTime}{November 2022\xspace}
\newcommand{\myVersion}{1.0}

% ********************************************************************
% Setup, finetuning, and useful commands
% ********************************************************************
\providecommand{\mLyX}{L\kern-.1667em\lower.25em\hbox{Y}\kern-.125emX\@}
\newcommand{\ie}{\textit{i.\,e.}}
\newcommand{\Ie}{\textit{I.\,e.}}
\newcommand{\eg}{\textit{e.\,g.}}
\newcommand{\Eg}{\textit{E.\,g.}}
\newcommand{\etc}{\textit{etc.}}
% ****************************************************************************************************


% ****************************************************************************************************
% 3. Loading some handy packages
% ****************************************************************************************************
% ********************************************************************
% Packages with options that might require adjustments
% ********************************************************************
\PassOptionsToPackage{ngerman,american}{babel} % change this to your language(s), main language last
% Spanish languages need extra options in order to work with this template
%\PassOptionsToPackage{spanish,es-lcroman}{babel}
\usepackage{babel}

\usepackage{csquotes}
\PassOptionsToPackage{%
  backend=biber,bibencoding=utf8, %instead of bibtex
  %backend=bibtex8,bibencoding=ascii,%
  language=auto,%
  style=numeric-comp,%
  %style=authoryear-comp, % Author 1999, 2010
  %bibstyle=authoryear,dashed=false, % dashed: substitute rep. author with ---
  sorting=nyt, % name, year, title
  maxbibnames=10, % default: 3, et al.
  %backref=true,%
  natbib=true % natbib compatibility mode (\citep and \citet still work)
}{biblatex}
\usepackage{biblatex}

\PassOptionsToPackage{fleqn}{amsmath}       % math environments and more by the AMS
\usepackage{amsmath}

% ********************************************************************
% General useful packages
% ********************************************************************
\usepackage{graphicx} %
\usepackage{scrhack} % fix warnings when using KOMA with listings package
\usepackage{xspace} % to get the spacing after macros right
\PassOptionsToPackage{printonlyused,smaller}{acronym}
\usepackage{acronym} % nice macros for handling all acronyms in the thesis
%\renewcommand{\bflabel}[1]{{#1}\hfill} % fix the list of acronyms --> no longer working
%\renewcommand*{\acsfont}[1]{\textsc{#1}}
%\renewcommand*{\aclabelfont}[1]{\acsfont{#1}}
%\def\bflabel#1{{#1\hfill}}
\def\bflabel#1{{\acsfont{#1}\hfill}}
\def\aclabelfont#1{\acsfont{#1}}
% ****************************************************************************************************
%\usepackage{pgfplots} % External TikZ/PGF support (thanks to Andreas Nautsch)
%\usetikzlibrary{external}
%\tikzexternalize[mode=list and make, prefix=ext-tikz/]
% ****************************************************************************************************


% ****************************************************************************************************
% 4. Setup floats: tables, (sub)figures, and captions
% ****************************************************************************************************
\usepackage{tabularx} % better tables
\setlength{\extrarowheight}{3pt} % increase table row height
\newcommand{\tableheadline}[1]{\multicolumn{1}{l}{\spacedlowsmallcaps{#1}}}
\newcommand{\myfloatalign}{\centering} % to be used with each float for alignment
\usepackage{subfig}
% ****************************************************************************************************


% ****************************************************************************************************
% 5. Setup code listings
% ****************************************************************************************************
\usepackage{listings}
%\lstset{emph={trueIndex,root},emphstyle=\color{BlueViolet}}%\underbar} % for special keywords
\lstset{language=[LaTeX]Tex,%C++,
  morekeywords={PassOptionsToPackage,selectlanguage},
  keywordstyle=\color{RoyalBlue},%\bfseries,
  basicstyle=\small\ttfamily,
  %identifierstyle=\color{NavyBlue},
  commentstyle=\color{Green}\ttfamily,
  stringstyle=\rmfamily,
  numbers=none,%left,%
  numberstyle=\scriptsize,%\tiny
  stepnumber=5,
  numbersep=8pt,
  showstringspaces=false,
  breaklines=true,
  %frameround=ftff,
  %frame=single,
  belowcaptionskip=.75\baselineskip
  %frame=L
}
% ****************************************************************************************************


\PassOptionsToPackage{margincaption,outercaption,ragged,wide,centerbody}{sidecap}
\RequirePackage{sidecap}
\sidecaptionvpos{figure}{t}
\sidecaptionvpos{table}{t}

\DeclareCaptionLabelFormat{slsc}{\spacedlowsmallcaps{#1 #2}}
\DeclareCaptionLabelSeparator{spacednewline}{\\[1ex]}
\captionsetup[SCfigure]{format=plain,labelsep=spacednewline,labelfont={rm,small},labelformat=slsc,textfont=rm,font=footnotesize,singlelinecheck=true,position=top}
\captionsetup[SCtable]{format=plain,labelsep=spacednewline,labelfont={rm,small},labelformat=slsc,textfont=rm,font=footnotesize,singlelinecheck=true,position=top}

% ****************************************************************************************************
% 6. Last calls before the bar closes
% ****************************************************************************************************
% ********************************************************************
% Her Majesty herself
% ********************************************************************
\usepackage{classicthesis}


% ********************************************************************
% Fine-tune hyperreferences (hyperref should be called last)
% ********************************************************************
\hypersetup{%
  %draft, % hyperref's draft mode, for printing see below
  colorlinks=true, linktocpage=true, pdfstartpage=3, pdfstartview=FitV,%
  % uncomment the following line if you want to have black links (e.g., for printing)
  %colorlinks=false, linktocpage=false, pdfstartpage=3, pdfstartview=FitV, pdfborder={0 0 0},%
  breaklinks=true, pageanchor=true,%
  pdfpagemode=UseNone, %
  % pdfpagemode=UseOutlines,%
  plainpages=false, bookmarksnumbered, bookmarksopen=true, bookmarksopenlevel=1,%
  hypertexnames=true, pdfhighlight=/O,%nesting=true,%frenchlinks,%
  urlcolor=CTurl, linkcolor=CTlink, citecolor=CTcitation, %pagecolor=RoyalBlue,%
  %urlcolor=Black, linkcolor=Black, citecolor=Black, %pagecolor=Black,%
  pdftitle={\myTitle},%
  pdfauthor={\textcopyright\ \myName, \myUni},%
  pdfsubject={},%
  pdfkeywords={},%
  pdfcreator={pdfLaTeX},%
  pdfproducer={LaTeX with hyperref and classicthesis}%
}


% ********************************************************************
% Setup autoreferences (hyperref and babel)
% ********************************************************************
% There are some issues regarding autorefnames
% http://www.tex.ac.uk/cgi-bin/texfaq2html?label=latexwords
% you have to redefine the macros for the
% language you use, e.g., american, ngerman
% (as chosen when loading babel/AtBeginDocument)
% ********************************************************************
\makeatletter
\@ifpackageloaded{babel}%
{%
  \addto\extrasamerican{%
    \renewcommand*{\figureautorefname}{Figure}%
    \renewcommand*{\tableautorefname}{Table}%
    \renewcommand*{\partautorefname}{Part}%
    \renewcommand*{\chapterautorefname}{Chapter}%
    \renewcommand*{\sectionautorefname}{Section}%
    \renewcommand*{\subsectionautorefname}{Section}%
    \renewcommand*{\subsubsectionautorefname}{Section}%
  }%
  \addto\extrasngerman{%
    \renewcommand*{\paragraphautorefname}{Absatz}%
    \renewcommand*{\subparagraphautorefname}{Unterabsatz}%
    \renewcommand*{\footnoteautorefname}{Fu\"snote}%
    \renewcommand*{\FancyVerbLineautorefname}{Zeile}%
    \renewcommand*{\theoremautorefname}{Theorem}%
    \renewcommand*{\appendixautorefname}{Anhang}%
    \renewcommand*{\equationautorefname}{Gleichung}%
    \renewcommand*{\itemautorefname}{Punkt}%
  }%
  % Fix to getting autorefs for subfigures right (thanks to Belinda Vogt for changing the definition)
  \providecommand{\subfigureautorefname}{\figureautorefname}%
}{\relax}
\makeatother


% ********************************************************************
% Development Stuff
% ********************************************************************
\listfiles
%\PassOptionsToPackage{l2tabu,orthodox,abort}{nag}
%  \usepackage{nag}
%\PassOptionsToPackage{warning, all}{onlyamsmath}
%  \usepackage{onlyamsmath}


% ****************************************************************************************************
% 7. Further adjustments (experimental)
% ****************************************************************************************************
% ********************************************************************
% Changing the text area
% ********************************************************************

% Palatino  10pt: 288--312pt | 609--657pt
%\areaset[current]{312pt}{761pt} % 686 (factor 2.2) + 33 head + 42 head \the\footskip
\areaset[current]{312pt}{657pt} % 686 (factor 2.2) + 33 head + 42 head \the\footskip
%\setlength{\marginparwidth}{7em}%
%\setlength{\marginparsep}{2em}%

% ********************************************************************
% Using different fonts
% ********************************************************************
%\usepackage[oldstylenums]{kpfonts} % oldstyle notextcomp
\usepackage[osf,tt=false]{libertine}
%\usepackage[light,condensed,math]{iwona}
%\renewcommand{\sfdefault}{iwona}
%\usepackage{lmodern} % <-- no osf support :-(
%\usepackage{cfr-lm} %
%\usepackage[urw-garamond]{mathdesign} <-- no osf support :-(
%\usepackage[default,osfigures]{opensans} % scale=0.95
%\usepackage{FiraSans}
%\usepackage[opticals,mathlf]{MinionPro} % onlytext

% TODO use euler-digits (and in matplotlibrc)?
\usepackage[small]{eulervm}
% ********************************************************************
%\usepackage[largesc,osf]{newpxtext}
%\linespread{1.05} % a bit more for Palatino
% Used to fix these:
% https://bitbucket.org/amiede/classicthesis/issues/139/italics-in-pallatino-capitals-chapter
% https://bitbucket.org/amiede/classicthesis/issues/45/problema-testatine-su-classicthesis-style
% ********************************************************************
% ****************************************************************************************************




\usepackage[letterpaper, inner=1.5in, outer=3in, top=1.25in, bottom=1.25in]{geometry}

\setlength{\marginparwidth}{11em}%
\setlength{\marginparsep}{2.5em}%


\edef\sidecaptionsep{\the\marginparsep}
\usepackage{sidenotes}



\newcommand{\etal}{\emph{et~al.}\xspace}
\newcommand{\unix}{\textsc{Unix}\xspace}
\newcommand{\Twizzler}{Twizzler\xspace}
\newcommand{\NVM}{NVM\xspace}

\newcommand{\unixkv}{\texttt{unixkv}\xspace}
\newcommand{\nvkv}{\texttt{twzkv}\xspace}
\newcommand{\ramrbt}{\texttt{ramrbt}\xspace}
\newcommand{\unixrbt}{\texttt{unixrbt}\xspace}
\newcommand{\nvrbt}{\texttt{twzrbt}\xspace}

\newcommand{\dab}[1]{{\textcolor{cyan}{[[#1 -- dab]]}}}


\newcommand{\unedit}[1]{{\leavevmode\color{red}#1}}

%%%%%%
% TODO: get rid of these
\newcommand{\observe}[2]{{\leavevmode\color{red}#2}}
\newcommand{\observation}[1]{{\leavevmode\color{red}#1}}
\newcommand{\observations}[1]{{\leavevmode\color{red}#1}}
\makeatletter
\newcommand\footnoteref[1]{\protected@xdef\@thefnmark{\ref{#1}}\@footnotemark}
\makeatother
%%%%%%

\newcommand{\libcore}{\texttt{libtwz}\xspace}

\usepackage{siunitx}
\usepackage{amsmath}
\usepackage{rotating}
\usepackage[pdf]{graphviz}
\usepackage{todonotes}
\usepackage{annotate-equations}
% ****************************************************************************************************
% If you like the classicthesis, then I would appreciate a postcard.
% My address can be found in the file ClassicThesis.pdf. A collection
% of the postcards I received so far is available online at
% http://postcards.miede.de
% ****************************************************************************************************


% ****************************************************************************************************
% 0. Set the encoding of your files. UTF-8 is the only sensible encoding nowadays. If you can't read
% äöüßáéçèê∂åëæƒÏ€ then change the encoding setting in your editor, not the line below. If your editor
% does not support utf8 use another editor!
% ****************************************************************************************************
\PassOptionsToPackage{utf8}{inputenc}
\usepackage{inputenc}

\PassOptionsToPackage{T1}{fontenc} % T2A for cyrillics
\usepackage{fontenc}


% ****************************************************************************************************
% 1. Configure classicthesis for your needs here, e.g., remove "drafting" below
% in order to deactivate the time-stamp on the pages
% (see ClassicThesis.pdf for more information):
% ****************************************************************************************************
\PassOptionsToPackage{
  drafting=true,    % print version information on the bottom of the pages
  tocaligned=false, % the left column of the toc will be aligned (no indentation)
  dottedtoc=true,  % page numbers in ToC flushed right
  eulerchapternumbers=true, % use AMS Euler for chapter font (otherwise Palatino)
  linedheaders=false,       % chaper headers will have line above and beneath
  floatperchapter=true,     % numbering per chapter for all floats (i.e., Figure 1.1)
  eulermath=true,  % use awesome Euler fonts for mathematical formulae (only with pdfLaTeX)
  beramono=true,    % toggle a nice monospaced font (w/ bold)
  palatino=false,    % deactivate standard font for loading another one, see the last section at the end of this file for suggestions
  style=classicthesis % classicthesis, arsclassica
}{classicthesis}


% ****************************************************************************************************
% 2. Personal data and user ad-hoc commands (insert your own data here)
% ****************************************************************************************************
\newcommand{\myTitle}{Operating Systems for Far Out Memories\xspace}
\newcommand{\mySubtitle}{Building a new OS for Upcoming Hardware\xspace}
%\newcommand{\myDegree}{Doktor-Ingenieur (Dr.-Ing.)\xspace}
\newcommand{\myName}{Daniel Bittman\xspace}
%\newcommand{\myProf}{Put name here\xspace}
%\newcommand{\myOtherProf}{Put name here\xspace}
%\newcommand{\mySupervisor}{Put name here\xspace}
%\newcommand{\myFaculty}{Put data here\xspace}
\newcommand{\myDepartment}{Computer Science\xspace}
\newcommand{\myUni}{University of California, Santa Cruz\xspace}
\newcommand{\myLocation}{Santa Cruz, CA\xspace}
\newcommand{\myTime}{November 2022\xspace}
\newcommand{\myVersion}{1.0}

% ********************************************************************
% Setup, finetuning, and useful commands
% ********************************************************************
\providecommand{\mLyX}{L\kern-.1667em\lower.25em\hbox{Y}\kern-.125emX\@}
\newcommand{\ie}{\textit{i.\,e.}}
\newcommand{\Ie}{\textit{I.\,e.}}
\newcommand{\eg}{\textit{e.\,g.}}
\newcommand{\Eg}{\textit{E.\,g.}}
\newcommand{\etc}{\textit{etc.}}
% ****************************************************************************************************


% ****************************************************************************************************
% 3. Loading some handy packages
% ****************************************************************************************************
% ********************************************************************
% Packages with options that might require adjustments
% ********************************************************************
\PassOptionsToPackage{ngerman,american}{babel} % change this to your language(s), main language last
% Spanish languages need extra options in order to work with this template
%\PassOptionsToPackage{spanish,es-lcroman}{babel}
\usepackage{babel}

\usepackage{csquotes}
\PassOptionsToPackage{%
  backend=biber,bibencoding=utf8, %instead of bibtex
  %backend=bibtex8,bibencoding=ascii,%
  language=auto,%
  style=numeric-comp,%
  %style=authoryear-comp, % Author 1999, 2010
  %bibstyle=authoryear,dashed=false, % dashed: substitute rep. author with ---
  sorting=nyt, % name, year, title
  maxbibnames=10, % default: 3, et al.
  %backref=true,%
  natbib=true % natbib compatibility mode (\citep and \citet still work)
}{biblatex}
\usepackage{biblatex}

\PassOptionsToPackage{fleqn}{amsmath}       % math environments and more by the AMS
\usepackage{amsmath}

% ********************************************************************
% General useful packages
% ********************************************************************
\usepackage{graphicx} %
\usepackage{scrhack} % fix warnings when using KOMA with listings package
\usepackage{xspace} % to get the spacing after macros right
\PassOptionsToPackage{printonlyused,smaller}{acronym}
\usepackage{acronym} % nice macros for handling all acronyms in the thesis
%\renewcommand{\bflabel}[1]{{#1}\hfill} % fix the list of acronyms --> no longer working
%\renewcommand*{\acsfont}[1]{\textsc{#1}}
%\renewcommand*{\aclabelfont}[1]{\acsfont{#1}}
%\def\bflabel#1{{#1\hfill}}
\def\bflabel#1{{\acsfont{#1}\hfill}}
\def\aclabelfont#1{\acsfont{#1}}
% ****************************************************************************************************
%\usepackage{pgfplots} % External TikZ/PGF support (thanks to Andreas Nautsch)
%\usetikzlibrary{external}
%\tikzexternalize[mode=list and make, prefix=ext-tikz/]
% ****************************************************************************************************


% ****************************************************************************************************
% 4. Setup floats: tables, (sub)figures, and captions
% ****************************************************************************************************
\usepackage{tabularx} % better tables
\setlength{\extrarowheight}{3pt} % increase table row height
\newcommand{\tableheadline}[1]{\multicolumn{1}{l}{\spacedlowsmallcaps{#1}}}
\newcommand{\myfloatalign}{\centering} % to be used with each float for alignment
\usepackage{subfig}
% ****************************************************************************************************


% ****************************************************************************************************
% 5. Setup code listings
% ****************************************************************************************************
\usepackage{listings}
%\lstset{emph={trueIndex,root},emphstyle=\color{BlueViolet}}%\underbar} % for special keywords
\lstset{language=[LaTeX]Tex,%C++,
  morekeywords={PassOptionsToPackage,selectlanguage},
  keywordstyle=\color{RoyalBlue},%\bfseries,
  basicstyle=\small\ttfamily,
  %identifierstyle=\color{NavyBlue},
  commentstyle=\color{Green}\ttfamily,
  stringstyle=\rmfamily,
  numbers=none,%left,%
  numberstyle=\scriptsize,%\tiny
  stepnumber=5,
  numbersep=8pt,
  showstringspaces=false,
  breaklines=true,
  %frameround=ftff,
  %frame=single,
  belowcaptionskip=.75\baselineskip
  %frame=L
}
% ****************************************************************************************************


\PassOptionsToPackage{margincaption,outercaption,ragged,wide,centerbody}{sidecap}
\RequirePackage{sidecap}
\sidecaptionvpos{figure}{t}
\sidecaptionvpos{table}{t}

\DeclareCaptionLabelFormat{slsc}{\spacedlowsmallcaps{#1 #2}}
\DeclareCaptionLabelSeparator{spacednewline}{\\[1ex]}
\captionsetup[SCfigure]{format=plain,labelsep=spacednewline,labelfont={rm,small},labelformat=slsc,textfont=rm,font=footnotesize,singlelinecheck=true,position=top}
\captionsetup[SCtable]{format=plain,labelsep=spacednewline,labelfont={rm,small},labelformat=slsc,textfont=rm,font=footnotesize,singlelinecheck=true,position=top}

% ****************************************************************************************************
% 6. Last calls before the bar closes
% ****************************************************************************************************
% ********************************************************************
% Her Majesty herself
% ********************************************************************
\usepackage{classicthesis}


% ********************************************************************
% Fine-tune hyperreferences (hyperref should be called last)
% ********************************************************************
\hypersetup{%
  %draft, % hyperref's draft mode, for printing see below
  colorlinks=true, linktocpage=true, pdfstartpage=3, pdfstartview=FitV,%
  % uncomment the following line if you want to have black links (e.g., for printing)
  %colorlinks=false, linktocpage=false, pdfstartpage=3, pdfstartview=FitV, pdfborder={0 0 0},%
  breaklinks=true, pageanchor=true,%
  pdfpagemode=UseNone, %
  % pdfpagemode=UseOutlines,%
  plainpages=false, bookmarksnumbered, bookmarksopen=true, bookmarksopenlevel=1,%
  hypertexnames=true, pdfhighlight=/O,%nesting=true,%frenchlinks,%
  urlcolor=CTurl, linkcolor=CTlink, citecolor=CTcitation, %pagecolor=RoyalBlue,%
  %urlcolor=Black, linkcolor=Black, citecolor=Black, %pagecolor=Black,%
  pdftitle={\myTitle},%
  pdfauthor={\textcopyright\ \myName, \myUni},%
  pdfsubject={},%
  pdfkeywords={},%
  pdfcreator={pdfLaTeX},%
  pdfproducer={LaTeX with hyperref and classicthesis}%
}


% ********************************************************************
% Setup autoreferences (hyperref and babel)
% ********************************************************************
% There are some issues regarding autorefnames
% http://www.tex.ac.uk/cgi-bin/texfaq2html?label=latexwords
% you have to redefine the macros for the
% language you use, e.g., american, ngerman
% (as chosen when loading babel/AtBeginDocument)
% ********************************************************************
\makeatletter
\@ifpackageloaded{babel}%
{%
  \addto\extrasamerican{%
    \renewcommand*{\figureautorefname}{Figure}%
    \renewcommand*{\tableautorefname}{Table}%
    \renewcommand*{\partautorefname}{Part}%
    \renewcommand*{\chapterautorefname}{Chapter}%
    \renewcommand*{\sectionautorefname}{Section}%
    \renewcommand*{\subsectionautorefname}{Section}%
    \renewcommand*{\subsubsectionautorefname}{Section}%
  }%
  \addto\extrasngerman{%
    \renewcommand*{\paragraphautorefname}{Absatz}%
    \renewcommand*{\subparagraphautorefname}{Unterabsatz}%
    \renewcommand*{\footnoteautorefname}{Fu\"snote}%
    \renewcommand*{\FancyVerbLineautorefname}{Zeile}%
    \renewcommand*{\theoremautorefname}{Theorem}%
    \renewcommand*{\appendixautorefname}{Anhang}%
    \renewcommand*{\equationautorefname}{Gleichung}%
    \renewcommand*{\itemautorefname}{Punkt}%
  }%
  % Fix to getting autorefs for subfigures right (thanks to Belinda Vogt for changing the definition)
  \providecommand{\subfigureautorefname}{\figureautorefname}%
}{\relax}
\makeatother


% ********************************************************************
% Development Stuff
% ********************************************************************
\listfiles
%\PassOptionsToPackage{l2tabu,orthodox,abort}{nag}
%  \usepackage{nag}
%\PassOptionsToPackage{warning, all}{onlyamsmath}
%  \usepackage{onlyamsmath}


% ****************************************************************************************************
% 7. Further adjustments (experimental)
% ****************************************************************************************************
% ********************************************************************
% Changing the text area
% ********************************************************************

% Palatino  10pt: 288--312pt | 609--657pt
%\areaset[current]{312pt}{761pt} % 686 (factor 2.2) + 33 head + 42 head \the\footskip
\areaset[current]{312pt}{657pt} % 686 (factor 2.2) + 33 head + 42 head \the\footskip
%\setlength{\marginparwidth}{7em}%
%\setlength{\marginparsep}{2em}%

% ********************************************************************
% Using different fonts
% ********************************************************************
%\usepackage[oldstylenums]{kpfonts} % oldstyle notextcomp
\usepackage[osf,tt=false]{libertine}
%\usepackage[light,condensed,math]{iwona}
%\renewcommand{\sfdefault}{iwona}
%\usepackage{lmodern} % <-- no osf support :-(
%\usepackage{cfr-lm} %
%\usepackage[urw-garamond]{mathdesign} <-- no osf support :-(
%\usepackage[default,osfigures]{opensans} % scale=0.95
%\usepackage{FiraSans}
%\usepackage[opticals,mathlf]{MinionPro} % onlytext

% TODO use euler-digits (and in matplotlibrc)?
\usepackage[small]{eulervm}
% ********************************************************************
%\usepackage[largesc,osf]{newpxtext}
%\linespread{1.05} % a bit more for Palatino
% Used to fix these:
% https://bitbucket.org/amiede/classicthesis/issues/139/italics-in-pallatino-capitals-chapter
% https://bitbucket.org/amiede/classicthesis/issues/45/problema-testatine-su-classicthesis-style
% ********************************************************************
% ****************************************************************************************************




\usepackage[letterpaper, inner=1.5in, outer=3in, top=1.25in, bottom=1.25in]{geometry}

\setlength{\marginparwidth}{11em}%
\setlength{\marginparsep}{2.5em}%


\edef\sidecaptionsep{\the\marginparsep}
\usepackage{sidenotes}



\newcommand{\etal}{\emph{et~al.}\xspace}
\newcommand{\unix}{\textsc{Unix}\xspace}
\newcommand{\Twizzler}{Twizzler\xspace}
\newcommand{\NVM}{NVM\xspace}

\newcommand{\unixkv}{\texttt{unixkv}\xspace}
\newcommand{\nvkv}{\texttt{twzkv}\xspace}
\newcommand{\ramrbt}{\texttt{ramrbt}\xspace}
\newcommand{\unixrbt}{\texttt{unixrbt}\xspace}
\newcommand{\nvrbt}{\texttt{twzrbt}\xspace}

\newcommand{\dab}[1]{{\textcolor{cyan}{[[#1 -- dab]]}}}


\newcommand{\unedit}[1]{{\leavevmode\color{red}#1}}

%%%%%%
% TODO: get rid of these
\newcommand{\observe}[2]{{\leavevmode\color{red}#2}}
\newcommand{\observation}[1]{{\leavevmode\color{red}#1}}
\newcommand{\observations}[1]{{\leavevmode\color{red}#1}}
\makeatletter
\newcommand\footnoteref[1]{\protected@xdef\@thefnmark{\ref{#1}}\@footnotemark}
\makeatother
%%%%%%

\newcommand{\libcore}{\texttt{libtwz}\xspace}

\usepackage{siunitx}
\usepackage{amsmath}
\usepackage{rotating}
\usepackage[pdf]{graphviz}
\usepackage{todonotes}
\usepackage{annotate-equations}
% ****************************************************************************************************
% If you like the classicthesis, then I would appreciate a postcard.
% My address can be found in the file ClassicThesis.pdf. A collection
% of the postcards I received so far is available online at
% http://postcards.miede.de
% ****************************************************************************************************


% ****************************************************************************************************
% 0. Set the encoding of your files. UTF-8 is the only sensible encoding nowadays. If you can't read
% äöüßáéçèê∂åëæƒÏ€ then change the encoding setting in your editor, not the line below. If your editor
% does not support utf8 use another editor!
% ****************************************************************************************************
\PassOptionsToPackage{utf8}{inputenc}
\usepackage{inputenc}

\PassOptionsToPackage{T1}{fontenc} % T2A for cyrillics
\usepackage{fontenc}


% ****************************************************************************************************
% 1. Configure classicthesis for your needs here, e.g., remove "drafting" below
% in order to deactivate the time-stamp on the pages
% (see ClassicThesis.pdf for more information):
% ****************************************************************************************************
\PassOptionsToPackage{
  drafting=true,    % print version information on the bottom of the pages
  tocaligned=false, % the left column of the toc will be aligned (no indentation)
  dottedtoc=true,  % page numbers in ToC flushed right
  eulerchapternumbers=true, % use AMS Euler for chapter font (otherwise Palatino)
  linedheaders=false,       % chaper headers will have line above and beneath
  floatperchapter=true,     % numbering per chapter for all floats (i.e., Figure 1.1)
  eulermath=true,  % use awesome Euler fonts for mathematical formulae (only with pdfLaTeX)
  beramono=true,    % toggle a nice monospaced font (w/ bold)
  palatino=false,    % deactivate standard font for loading another one, see the last section at the end of this file for suggestions
  style=classicthesis % classicthesis, arsclassica
}{classicthesis}


% ****************************************************************************************************
% 2. Personal data and user ad-hoc commands (insert your own data here)
% ****************************************************************************************************
\newcommand{\myTitle}{Operating Systems for Far Out Memories\xspace}
\newcommand{\mySubtitle}{Building a new OS for Upcoming Hardware\xspace}
%\newcommand{\myDegree}{Doktor-Ingenieur (Dr.-Ing.)\xspace}
\newcommand{\myName}{Daniel Bittman\xspace}
%\newcommand{\myProf}{Put name here\xspace}
%\newcommand{\myOtherProf}{Put name here\xspace}
%\newcommand{\mySupervisor}{Put name here\xspace}
%\newcommand{\myFaculty}{Put data here\xspace}
\newcommand{\myDepartment}{Computer Science\xspace}
\newcommand{\myUni}{University of California, Santa Cruz\xspace}
\newcommand{\myLocation}{Santa Cruz, CA\xspace}
\newcommand{\myTime}{November 2022\xspace}
\newcommand{\myVersion}{1.0}

% ********************************************************************
% Setup, finetuning, and useful commands
% ********************************************************************
\providecommand{\mLyX}{L\kern-.1667em\lower.25em\hbox{Y}\kern-.125emX\@}
\newcommand{\ie}{\textit{i.\,e.}}
\newcommand{\Ie}{\textit{I.\,e.}}
\newcommand{\eg}{\textit{e.\,g.}}
\newcommand{\Eg}{\textit{E.\,g.}}
\newcommand{\etc}{\textit{etc}}
% ****************************************************************************************************


% ****************************************************************************************************
% 3. Loading some handy packages
% ****************************************************************************************************
% ********************************************************************
% Packages with options that might require adjustments
% ********************************************************************
\PassOptionsToPackage{ngerman,american}{babel} % change this to your language(s), main language last
% Spanish languages need extra options in order to work with this template
%\PassOptionsToPackage{spanish,es-lcroman}{babel}
\usepackage{babel}

\usepackage{csquotes}
\PassOptionsToPackage{%
  backend=biber,bibencoding=utf8, %instead of bibtex
  %backend=bibtex8,bibencoding=ascii,%
  language=auto,%
  style=numeric-comp,%
  %style=authoryear-comp, % Author 1999, 2010
  %bibstyle=authoryear,dashed=false, % dashed: substitute rep. author with ---
  sorting=nyt, % name, year, title
  maxbibnames=10, % default: 3, et al.
  %backref=true,%
  natbib=true % natbib compatibility mode (\citep and \citet still work)
}{biblatex}
\usepackage{biblatex}

\PassOptionsToPackage{fleqn}{amsmath}       % math environments and more by the AMS
\usepackage{amsmath}

% ********************************************************************
% General useful packages
% ********************************************************************
\usepackage{graphicx} %
\usepackage{scrhack} % fix warnings when using KOMA with listings package
\usepackage{xspace} % to get the spacing after macros right
\PassOptionsToPackage{printonlyused,smaller}{acronym}
\usepackage{acronym} % nice macros for handling all acronyms in the thesis
%\renewcommand{\bflabel}[1]{{#1}\hfill} % fix the list of acronyms --> no longer working
%\renewcommand*{\acsfont}[1]{\textsc{#1}}
%\renewcommand*{\aclabelfont}[1]{\acsfont{#1}}
%\def\bflabel#1{{#1\hfill}}
\def\bflabel#1{{\acsfont{#1}\hfill}}
\def\aclabelfont#1{\acsfont{#1}}
% ****************************************************************************************************
%\usepackage{pgfplots} % External TikZ/PGF support (thanks to Andreas Nautsch)
%\usetikzlibrary{external}
%\tikzexternalize[mode=list and make, prefix=ext-tikz/]
% ****************************************************************************************************


% ****************************************************************************************************
% 4. Setup floats: tables, (sub)figures, and captions
% ****************************************************************************************************
\usepackage{tabularx} % better tables
\setlength{\extrarowheight}{3pt} % increase table row height
\newcommand{\tableheadline}[1]{\multicolumn{1}{l}{\spacedlowsmallcaps{#1}}}
\newcommand{\myfloatalign}{\centering} % to be used with each float for alignment
\usepackage{subfig}
% ****************************************************************************************************


% ****************************************************************************************************
% 5. Setup code listings
% ****************************************************************************************************
\usepackage{listings}
%\lstset{emph={trueIndex,root},emphstyle=\color{BlueViolet}}%\underbar} % for special keywords
\lstset{language=[LaTeX]Tex,%C++,
  morekeywords={PassOptionsToPackage,selectlanguage},
  keywordstyle=\color{RoyalBlue},%\bfseries,
  basicstyle=\small\ttfamily,
  %identifierstyle=\color{NavyBlue},
  commentstyle=\color{Green}\ttfamily,
  stringstyle=\rmfamily,
  numbers=none,%left,%
  numberstyle=\scriptsize,%\tiny
  stepnumber=5,
  numbersep=8pt,
  showstringspaces=false,
  breaklines=true,
  %frameround=ftff,
  %frame=single,
  belowcaptionskip=.75\baselineskip
  %frame=L
}
% ****************************************************************************************************


\PassOptionsToPackage{margincaption,outercaption,ragged,wide,centerbody}{sidecap}
\RequirePackage{sidecap}
\sidecaptionvpos{figure}{t}
\sidecaptionvpos{table}{t}

\DeclareCaptionLabelFormat{slsc}{\spacedlowsmallcaps{#1 #2}}
\DeclareCaptionLabelSeparator{spacednewline}{\\[1ex]}
\captionsetup[SCfigure]{format=plain,labelsep=spacednewline,labelfont={rm,small},labelformat=slsc,textfont=rm,font=footnotesize,singlelinecheck=true,position=top}
\captionsetup[SCtable]{format=plain,labelsep=spacednewline,labelfont={rm,small},labelformat=slsc,textfont=rm,font=footnotesize,singlelinecheck=true,position=top}

% ****************************************************************************************************
% 6. Last calls before the bar closes
% ****************************************************************************************************
% ********************************************************************
% Her Majesty herself
% ********************************************************************
\usepackage{classicthesis}


% ********************************************************************
% Fine-tune hyperreferences (hyperref should be called last)
% ********************************************************************
\hypersetup{%
  %draft, % hyperref's draft mode, for printing see below
  colorlinks=true, linktocpage=true, pdfstartpage=3, pdfstartview=FitV,%
  % uncomment the following line if you want to have black links (e.g., for printing)
  %colorlinks=false, linktocpage=false, pdfstartpage=3, pdfstartview=FitV, pdfborder={0 0 0},%
  breaklinks=true, pageanchor=true,%
  pdfpagemode=UseNone, %
  % pdfpagemode=UseOutlines,%
  plainpages=false, bookmarksnumbered, bookmarksopen=true, bookmarksopenlevel=1,%
  hypertexnames=true, pdfhighlight=/O,%nesting=true,%frenchlinks,%
  urlcolor=CTurl, linkcolor=CTlink, citecolor=CTcitation, %pagecolor=RoyalBlue,%
  %urlcolor=Black, linkcolor=Black, citecolor=Black, %pagecolor=Black,%
  pdftitle={\myTitle},%
  pdfauthor={\textcopyright\ \myName, \myUni},%
  pdfsubject={},%
  pdfkeywords={},%
  pdfcreator={pdfLaTeX},%
  pdfproducer={LaTeX with hyperref and classicthesis}%
}


% ********************************************************************
% Setup autoreferences (hyperref and babel)
% ********************************************************************
% There are some issues regarding autorefnames
% http://www.tex.ac.uk/cgi-bin/texfaq2html?label=latexwords
% you have to redefine the macros for the
% language you use, e.g., american, ngerman
% (as chosen when loading babel/AtBeginDocument)
% ********************************************************************
\makeatletter
\@ifpackageloaded{babel}%
{%
  \addto\extrasamerican{%
    \renewcommand*{\figureautorefname}{Figure}%
    \renewcommand*{\tableautorefname}{Table}%
    \renewcommand*{\partautorefname}{Part}%
    \renewcommand*{\chapterautorefname}{Chapter}%
    \renewcommand*{\sectionautorefname}{Section}%
    \renewcommand*{\subsectionautorefname}{Section}%
    \renewcommand*{\subsubsectionautorefname}{Section}%
  }%
  \addto\extrasngerman{%
    \renewcommand*{\paragraphautorefname}{Absatz}%
    \renewcommand*{\subparagraphautorefname}{Unterabsatz}%
    \renewcommand*{\footnoteautorefname}{Fu\"snote}%
    \renewcommand*{\FancyVerbLineautorefname}{Zeile}%
    \renewcommand*{\theoremautorefname}{Theorem}%
    \renewcommand*{\appendixautorefname}{Anhang}%
    \renewcommand*{\equationautorefname}{Gleichung}%
    \renewcommand*{\itemautorefname}{Punkt}%
  }%
  % Fix to getting autorefs for subfigures right (thanks to Belinda Vogt for changing the definition)
  \providecommand{\subfigureautorefname}{\figureautorefname}%
}{\relax}
\makeatother


% ********************************************************************
% Development Stuff
% ********************************************************************
\listfiles
%\PassOptionsToPackage{l2tabu,orthodox,abort}{nag}
%  \usepackage{nag}
%\PassOptionsToPackage{warning, all}{onlyamsmath}
%  \usepackage{onlyamsmath}


% ****************************************************************************************************
% 7. Further adjustments (experimental)
% ****************************************************************************************************
% ********************************************************************
% Changing the text area
% ********************************************************************

% Palatino  10pt: 288--312pt | 609--657pt
%\areaset[current]{312pt}{761pt} % 686 (factor 2.2) + 33 head + 42 head \the\footskip
\areaset[current]{312pt}{657pt} % 686 (factor 2.2) + 33 head + 42 head \the\footskip
%\setlength{\marginparwidth}{7em}%
%\setlength{\marginparsep}{2em}%

% ********************************************************************
% Using different fonts
% ********************************************************************
%\usepackage[oldstylenums]{kpfonts} % oldstyle notextcomp
% TODO should we use osf?
%\usepackage[osf]{libertinus}
%\usepackage[light,condensed,math]{iwona}
%\renewcommand{\sfdefault}{iwona}
%\usepackage{lmodern} % <-- no osf support :-(
%\usepackage{cfr-lm} %
%\usepackage[urw-garamond]{mathdesign} <-- no osf support :-(
%\usepackage[default,osfigures]{opensans} % scale=0.95
%\usepackage{FiraSans}
%\usepackage[opticals,mathlf]{MinionPro} % onlytext
%\usepackage{crimson}
\setmainfont[Numbers={OldStyle,Proportional}]{Minion Pro}

%\usepackage[libertine]{newtxmath}


% TODO use euler-digits (and in matplotlibrc)?
\iffalse
  \usepackage{unicode-math}
  \unimathsetup{math-style=ISO, partial=upright, nabla=upright}

  \defaultfontfeatures{Scale=MatchLowercase}

  \setmathfont{Asana Math}
  \setmathfont[range={"0000-"0001,"0020-"007E,
        "00A0,"00A7-"00A8,"00AC,"00AF,"00B1,"00B4-"00B5,"00B7,
        "00D7,"00F7,
        "0131,
        "0237,"02C6-"02C7,"02D8-"02DA,"02DC,
        "0300-"030C,"030F,"0311,"0323-"0325,"032E-"0332,"0338,
        "0391-"0393,"0395-"03A1,"03A3-"03A8,"03B1-"03BB,
        "03BD-"03C1,"03C3-"03C9,"03D1,"03D5-"03D6,"03F5,
        "2016,"2018-"2019,"2021,"2026-"202C,"2032-"2037,"2044,
        "2057,"20D6-"20D7,"20DB-"20DD,"20E1,"20EE-"20EF,
        "210B-"210C,"210E-"2113,"2118,"211B-"211C,"2126-"2128,
        "212C-"212D,"2130-"2131,"2133,"2135,"2190-"2199,
        "21A4,"21A6,"21A9-"21AA,"21BC-"21CC,"21D0-"21D5,
        "2200,"2202-"2209,"220B-"220C,"220F-"2213,"2215-"221E,
        "2223,"2225,"2227-"222E,"2234-"2235,"2237-"223D,
        "2240-"224C,"2260-"2269,"226E-"2279,"2282-"228B,"228E,
        "2291-"2292,"2295-"2299,"22A2-"22A5,"22C0-"22C5,
        "22DC-"22DD,"22EF,"22F0-"22F1,
        "2308-"230B,"2320-"2321,"2329-"232A,"239B-"23AE,
        "23DC-"23DF,
        "27E8-"27E9,"27F5-"27FE,"2A0C,"2B1A,
        "1D400-"1D433,"1D49C,"1D49E-"1D49F,"1D4A2,"1D4A5-"1D4A6,
        "1D4A9-"1D4AC,"1D4AE-"1D4B5,"1D4D0-"1D4E9,"1D504-"1D505,
        "1D507-"1D50A,"1D50D-"1D514,"1D516-"1D51C,"1D51E-"1D537,
        "1D56C-"1D59F,"1D6A8-"1D6B8,"1D6BA-"1D6D2,"1D6D4-"1D6DD,
        "1D6DF,"1D6E1,"1D7CE-"1D7D7
      }]{Neo Euler}
  \setmathfont[range=up/{greek,Greek}, script-features={}, sscript-features={}
  ]{Neo Euler}
  \setmathfont[range=up/{latin,Latin}, script-features={}, sscript-features={}
  ]{Neo Euler}
  \setmathfont[range={bfup/{latin, Latin, greek, Greek}, frak, bffrak, cal},
  script-features={}, sscript-features={}
  ]{Neo Euler}
  \setmathfont[range={up/num, bfup/num, it, bfit, scr, bfscr,
        sfup, sfit, bfsfup, bfsfit, tt}
  ]{Neo Euler}
  \setmathfont[range={up/num, bfup/num}
  ]{Asana Math}
  \setmathfont[range=bfcal, Scale=MatchUppercase, Alternate]{Asana Math}
\fi
% ********************************************************************
%\usepackage[largesc,osf]{newpxtext}
%\linespread{1.05} % a bit more for Palatino
% Used to fix these:
% https://bitbucket.org/amiede/classicthesis/issues/139/italics-in-pallatino-capitals-chapter
% https://bitbucket.org/amiede/classicthesis/issues/45/problema-testatine-su-classicthesis-style
% ********************************************************************
% ****************************************************************************************************




\usepackage[letterpaper, inner=1.5in, outer=3in, top=1.25in, bottom=1.25in]{geometry}

\setlength{\marginparwidth}{11em}%
\setlength{\marginparsep}{2.5em}%


\edef\sidecaptionsep{\the\marginparsep}
\usepackage{sidenotes}



\newcommand{\etal}{\emph{et~al.}\xspace}
\newcommand{\unix}{\textsc{Unix}\xspace}
\newcommand{\Twizzler}{Twizzler\xspace}
\newcommand{\NVM}{\ac{nvm}\xspace}

\newcommand{\unixkv}{\texttt{unixkv}\xspace}
\newcommand{\nvkv}{\texttt{twzkv}\xspace}
\newcommand{\ramrbt}{\texttt{ramrbt}\xspace}
\newcommand{\unixrbt}{\texttt{unixrbt}\xspace}
\newcommand{\nvrbt}{\texttt{twzrbt}\xspace}

\newcommand{\dab}[1]{{\textcolor{cyan}{[[#1 -- dab]]}}}


\newcommand{\unedit}[1]{{\leavevmode\color{red}#1}}

%%%%%%
% TODO: get rid of these
\newcommand{\observe}[2]{{\leavevmode\color{red}#2}}
\newcommand{\observation}[1]{{\leavevmode\color{red}#1}}
\newcommand{\observations}[1]{{\leavevmode\color{red}#1}}
\makeatletter
\newcommand\footnoteref[1]{\protected@xdef\@thefnmark{\ref{#1}}\@footnotemark}
\makeatother
%%%%%%

\newcommand{\libcore}{\texttt{libtwz}\xspace}

\usepackage{siunitx}
\usepackage{amsmath}
\usepackage{rotating}
\usepackage{todonotes}
\usepackage{annotate-equations}
\renewcommand{\eqnannotationtext}[1]{\rmfamily\footnotesize#1\strut}
\renewcommand{\eqnhighlightheight}{}

\usepackage{setspace}
\setstretch{1.125}


%% TODO try different options for acronyms
\makeatletter
\AtBeginDocument{%
  \renewcommand*{\AC@hyperlink}[2]{%
    \begingroup
    \hypersetup{hidelinks}%
    \hyperlink{#1}{#2}%
    \endgroup
  }%
}
\makeatother
\renewcommand*{\acsfont}[1]{{#1}}

\newenvironment{chabstract}{\itshape\paragraph{Synopsis}}{\chendsep}

\usepackage{tikz}
\newcommand{\chendsep}
{%
  \begin{center}\begin{tikzpicture}[thick]
      \draw (-1,0) -- (1,0);
      \fill[rotate=45] (-0.075,-0.075) rectangle (0.075,0.075);
    \end{tikzpicture}\end{center}
}


\newenvironment{chconc}{\chendsep\rightskip1in\itshape\paragraph{Conclusion}}{}