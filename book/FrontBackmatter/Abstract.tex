%*******************************************************
% Abstract
%*******************************************************
%\renewcommand{\abstractname}{Abstract}
\pdfbookmark[1]{Abstract}{Abstract}
% \addcontentsline{toc}{chapter}{\tocEntry{Abstract}}
\begingroup
\let\clearpage\relax
\let\cleardoublepage\relax
\let\cleardoublepage\relax

\chapter*{Abstract}

Operating systems are built and designed around two driving forces: the capabilities of hardware, and the demands of
software. Yet traditional\sidenote{Read: old.} operating systems and programming models have inertia, resulting
in new hardware being shoehorned into old interfaces. As a result, programmers are limited in their ability to express
the important parts of their programs due to the layers of compatibility and overhead thrust upon them, despite their
persistent demands for higher throughput and lower latency. Operating system abstractions must evolve into the modern day.

We stand before an opportunity to study on how a confluence of trends may shift programming models away from a
traditional, process-centric view point towards a \emph{data-centric} one, in which \emph{data} is the primary citizen
of the system. This opportunity arises from trends in hardware---the increasing speed of interconnect and a collapsing of
the memory hierarchy caused by remote memory and persistent memory---and software---as our programs demand further
scalability, distribution, and raw speed. Merely relying on decades old abstractions and incremental change will
relegate novel hardware of the last decade to a fate of access via interfaces designed for tape and spinning rust.

This dissertation presents an alternative, where we build a new programming model designed around a disconnect from the
traditional bifurcation of the memory hierarchy, one where we assume that data is fundamental and compute is ephemeral.
This data-centric approach reframes the goals of the operating system and enables us to re-imagine classic systems
programming techniques into a model that \emph{facilitates} data sharing instead of hindering\sidenote{\Eg making shared
    data difficult via complex
    persistence management, rigid RPC data models, serialization, \etc.} it. We will cover the motivation and hardware trends
that lead to our design, define a design space based on those trends, and finally discuss Twizzler, a point in that
design space that exemplifies the ideals we will discuss. We will evaluate Twizzler with case-studies that demonstrate
system behavior, and efficacy and useability of our programming models, done by building several new pieces of software
for Twizzler. These, along with ported larger applications, are used to demonstate the performance of Twizzler and its
programming model, often showing performance increase just due to simplification of software layers.

\vfill


\endgroup

\vfill
