
\section{Non-volatile Memory and Bit Flips}
\label{sec:background}

%%% Byte-addressible NVMs on mem bus soon / general design and info Non-volatile memory

Non-volatile memory technologies~\cite{burr:ibmjrd08} such as phase-change memory
(PCM)~\cite{lee_architecting_2009}, resistive RAM (RRAM, or
memristors)~\cite{strukov:nature08,shen:dtc11}, Ferroelectric RAM
(FeRAM)~\cite{fox:2001feram}, and spin-torque transfer RAM
(STT-RAM)~\cite{sttram}, among others, have the potential to fundamentally
change the design of devices, operating systems, and applications. Although
these technologies are starting to make their way into consumer
devices~\cite{intel3dxpoint} and embedded systems~\cite{shen:dtc11}, their full
potential will be seen when they replace or coexist with DRAM as
byte-addressable non-volatile memory (\NVM). Such a memory hierarchy will allow
the processor, and thus applications, to use load and store instructions to
update persistent state, bypassing the high-latency I/O operations of the
OS. However, power consumption, especially for write operations,
and device lifetime are more serious concerns for these technologies than
for existing memory technologies.

\subsection{Optimizing for Memory Technologies}

Data structures should be designed to exploit the advantages and mitigate the
disadvantages of the technologies on which they are deployed.  For example, data
structures for disks are block-oriented and favor sequential access, while those
designed for flash reduce writes, especially random writes, often by trading them for an increase in
random reads~\cite{colgrove:sigmod15}.  Prior data structures and programming
models for
NVM~\cite{xu:fast16,meza:weed13,greenan:hotdep07,volos:asplos11,coburn:asplos11,condit:sosp09}
have typically exploited its byte-addressability while mitigating the relatively
slow access times of most \NVM technologies.  However, in the case of
technologies such as PCM or RRAM, existing research ignores two critical
characteristics: asymmetric read/write power usage and the ability to avoid
rewriting individual bits that are unchanged by a
write~\cite{burr:ibmjrd08,yang:iscas07}.

For example, writes to PCM are done by melting a cell's worth of material with a
relatively high current and cooling it at two different rates, leaving the material
in either an amorphous or crystalline phase~\cite{raoux:ibmjrd08}.  These two
phases have different electrical resistance, each corresponding to a bit value
of zero or one.  The writing process takes much more energy than reading the
phase of the cell, which is done by sensing the cell's resistance with a
relatively low current. To save energy, the PCM controller can avoid writing to
a cell during a write if it already contains the desired
value~\cite{yang:iscas07}, meaning that the major component of the power
required by a write is proportional not to the number of bits (or words)
\emph{written}, but rather to the number of bits \emph{actually flipped} by the
write. Based on this observation, we should design data structures for \NVM to
minimize the number of bits flipped as the structures are modified and accessed
rather than simply reducing the number of writes, as is more commonly done.

\subsection{Power Consumption of PCM and DRAM}

While our research applies to any \NVM technology in which writes are expensive,
we focus on PCM because its power consumption figures are more readily
available. Figure~\ref{fig:flip_throughput} shows the estimated power consumption of 1~GB of DRAM and
PCM as a function of bits flipped per second, using power measurements from
prior studies of memory
systems~\cite{dhiman_pdram:_2009,lee_architecting_2009,xiangyu_dong_nvsim:_2012,qureshi_scalable_2009,Chen_rethinkingdatabase,bittman:nvmsa18}.
The number of writes to DRAM has little effect on overall power consumption
since the \emph{entire} DRAM must be periodically refreshed (read and
rewritten); refresh dominates, resulting in a high power requirement
regardless of the number of writes. In contrast, PCM requires no
``maintenance'' power, but needs a great deal more energy to write an
individual bit (\textasciitilde50\,pJ/b~\cite{bedeschi_8mb_2004}) compared to the
low overhead for writing a DRAM page
(\textasciitilde1\,pJ/b~\cite{lee_architecting_2009}).  The result is that power use
for DRAM is largely proportional to memory \emph{size}, while power consumption
for PCM is largely proportional to \emph{cell change rate}. The exact position of the cross-over
point in Figure~\ref{fig:flip_throughput} will be narrowed down as these devices become more common; many features of these devices,
including asymmetric write-zero and write-one costs, increased density of PCM over DRAM, and
decreasing feature sizes, will affect the trade-off point over time.

\begin{SCfigure}
	\centering
	\includegraphics[width=\linewidth]{genfig/power}
	\caption[Power use as a function of flips per second]{Power use as a function of flips per second~\cite{bittman:nvmsa18}.}
	\label{fig:flip_throughput}
\end{SCfigure}

Figure~\ref{fig:flip_throughput} demonstrates the need for data structures for
PCM to minimize cell writes.  Because the memory controller can minimize the
cost of ``writing'' a memory cell with the same value it already contains, the
primary concern for data structures in PCM is reducing the number of bit flips,
which the memory controller cannot easily eliminate.

Power consumption is particularly concerning for battery-operated Internet of Things (IoT) devices,
which may become a significant consumer of \NVM technologies to facilitate fast power-up and reduce
idle power consumption~\cite{Jayakumar2014powering,jayakumar2014quickrecall}. Devices that collect
large amounts of data and write frequently to \NVM may find power usage
increasing depending on access patterns. Thus, IoT devices may benefit significantly
from bit-flip-aware systems and data structures.


\subsection{Wear-out}

Another significant advantage to avoiding bit flips is reducing memory cell
wear-out. \NVM technologies typically have a maximum number of lifetime writes,
and fewer writes means a longer lifetime. However, by avoiding unnecessary overwrites,
the controller would introduce uneven wear \emph{within} \NVM words where some of the bits flip
more frequently than others due to biases of certain writes.
%However, techniques that minimize bit
%flips may cause uneven wear \emph{within} \NVM words, where the reduction in bit
%flips comes from fixing some of the bits while the others vary at the original
%rate.
For example, pointer overwrites may only alter the low-order bits, except
for the few that are zero because of structure alignment in memory, if the
pointers are to nearby regions. Thus, the middle bits in a 64-bit word may
wear out faster than the lowest and highest bits. While reducing bit-flips increases the
average lifetime of the cells in a word, it has the potential to exacerbate the uneven wear
problem since such techniques might increase the biases of certain writes.

Fortunately, we can take advantage of existing research in wear-leveling for \NVM that allows
the controller to spread out the cell updates within a given word.
While a full remapping layer similar to a flash translation layer is infeasible
for \NVM---the overhead would be too high---hardware techniques such as row
shifting~\cite{zhou:isca09:fixed}, content-aware bit shuffling~\cite{Han:cabs}, and
start-gap wear leveling~\cite{qureshi:micro09} may be able to mitigate biased
write patterns with low overhead. This would allow \NVM to leverage bit flip
reduction to reduce wear even if the result is that some bits are flipped more
frequently than others.  These techniques, implemented at the memory
controller level, can work in tandem with the techniques described in this appendix since they benefit
bit flip reduction and can distribute ``hot'' bits across a word, mitigating the biased write
patterns bit flip reduction techniques may introduce.
%from any reduction in bit flips and can tolerate increases in biased write patterns.


\subsection{Reducing Impact of Bit Flips in Non-volatile Memory}

Although bit flips in \NVM have been studied previously, much of that work has
focused on hardware encoding, which re-encodes cache lines to reduce bit flips,
but re-encoding has limited efficacy~\cite{flipnwrite,
	jacobvitz2013coset, seyedzadeh2016coset} because it must also store information
on \textit{which} encoding was used. While hardware techniques are worth
exploring, software techniques to reduce bit flips can be more effective because
they can leverage semantic knowledge available in the software but not
visible in the memory controller's limited view of single cache lines.

Chen \etal~\cite{Chen_rethinkingdatabase} evaluate data structures on \NVM and
argue that reducing bit flips is workload dependent and difficult to reason
about, so we should strive to reduce writes because writes are approximately
proportional to bit flips. We found that this is often \emph{not} the case---our prior experiments
revealed that bit flips were often \emph{not} proportional to writes, and we were able to
examine bit flips and optimize for them in an example data structure~\cite{bittman:nvmsa18}. These findings are further
corroborated by our experiments in Section~\ref{sec:evaluation}.

Since bit flips directly affect power consumption and wear, we
can study three separate aspects for bit flip reduction:
\begin{enumerate}
	\item \textbf{Data structure design}: Since data organization plays a large role
	      in the writes that make it to memory, we designed new data structures built
	      around the idea of \textit{pointer distance}~\cite{xorll} instead of storing
	      %% TODO(BLOCKING) review this
	      pointers directly. While \emph{data} writes themselves significantly affect bit flips, these writes
	      are often unavoidable (since the data must be written), while data \emph{structure} writes are more
	      easily optimized (as we see in existing \NVM data structure research).
	      Furthermore, data structures often require a significant number of updates
	      over time, while data is often written once (since we can reduce writes by updating pointers instead
	      of moving data). Thus the overall proportion of bit flips caused by data writes may drop over time as
	      data structures are updated.
	      %---after all, if the goal is to reduce writes and bit
	      %flips, it would be advantageous to avoid moving data and instead update pointers to the data.
	\item \textbf{Effects of program operation}: A common source of writes is
	      the stack, where return addresses, saved registers, and
	      register spills are written. Understanding how these writes affect bit flips
	      plays a critical role in recommendations for bit flip reduction for system designers.
	\item \textbf{Effects of caching layers}: Since writes must first go through the
	      cache, it is vital to understand how different caching layers and cache sizes
	      affect bit flips in memory. Complicating matters is the unique consistency
	      challenges of \NVM~\cite{coburn:asplos11,volos:asplos11,condit:sosp09}, wherein programs often flush cache-lines to main
	      memory more frequently than they otherwise would, use write-through caching, or
	      more complex, hardware-supported cache flushing protocols. These questions are evaluated in
	      Section~\ref{sec:exp_cache}.
\end{enumerate}

\if0
	Data structures are often constructed using pointers that describe the
	relationships between different components. These pointers are updated during
	data structure updates, often being updated several times (e.g. tree rotation).
	Given the frequency of these updates, a clear candidate for bit flip reduction
	is to reduce the bit flips caused by pointer writes. One method, which we study
	here, is learned from XOR linked lists (discussed below), and involves XOR-ing
	pointers together to reduce the number of stored pointers while also clearing
	many bits to zero.  Of course, this technique only applies to pointer-heavy data
	structures---pointer-light data structures, such as open-addressed hash tables,
	may not benefit from these techniques. However, some of the optimizations
	discussed below still apply.
\fi
