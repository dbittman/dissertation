\section{Discussion}
\label{sec:disc}

\paragraph{Software Bit Flip Reduction}

The data structures presented here are both old and new ideas. While not
algorithmically different from existing implementations (both \xrbt and \rbt use
the same, standard red-black tree algorithms), they present a new approach to
implementation with optimizations for bit flipping. This
has not been sufficiently studied before in the context of software optimization;
after all, there is no theoretical advance nor is there an overwhelming
practical advantage to these data structures outside of the bit flip reduction,
an optimization goal that is new with \NVM. However, keeping this in mind has
huge ramifications for data structures in persistent memory and applications for
new storage technologies, as it presents a whole new field of
study in optimization and practical data structure design. The goal is
not performance improvements; instead we strive to prolong the lifetime of expensive
memory devices while reducing power use, with at most a minor
performance cost. These improvements can be achieved without hardware
changes, meaning even savings of 10\% ($1.1\times$) or
less are worthwhile to implement because savings are cumulative.

These optimizations are not specific to PCM; any memory with a significant read/write
disparity and bit-level updates could benefit from this. The energy savings from bit flip
optimization will, of course, be technology-dependent, numbers for which will solidify as the
technologies are adopted. Our estimates of the linear relationship
between flips and power use (Figure~\ref{fig:flip_throughput}) indicate that, on PCM, the energy
savings will be roughly proportional to the bit flip savings since the difference between read and
write energy is so high.

Bit flips can and should be reasoned about directly. Not only is it possible to
do so, but the methods presented here are straightforward once this goal is in mind.
Furthermore, while reducing writes \textit{can} reduce bit flips, we have
confirmed that this is not \textit{always} true. XOR linked lists reduced bit
flips without affecting writes, while \xrbt reduced writes over \xrbtbig
at the cost of increasing bit flips. With stack frames, the biggest reduction in bit flips
corresponded with no change in writes, while the reduction of writes was
correlated with only a modest bit flip reduction.

The implications are far-reaching, especially when considering novel computation
models that include storing program state in \NVM. Writes to the stack also
affect bit flips, but these can be dramatically optimized. Compilers can
implement standardized stack frame layouts for register spills that
save many bit flips while remaining backwards compatible since
nothing in these optimizations breaks existing ABIs. Further research is
required to better study the effects of stack frame layout in larger programs,
and engineering work is needed to build these features into existing toolchains.

Of course, we must be cautious to optimize where it matters. While different
allocation sizes reduced bit flips relative to each other, the overall effect
was minor compared to the savings gained in other data structures. In fact, the
reduction in allocation size from 48 to 40 bytes in \xrbt actually
\textit{increased} bit flips in calls to \malloc, but this increase is
dwarfed by the savings from the XOR pointers. Additionally, the hash table
saw a relatively small saving compared to other data structures since it
already flipped a minimal number of bits in the average case;
red-black trees often do more work during \textit{each} update operation,
resulting in a number of pointer updates. Hash tables often do their
``rebalancing'' during a single rehash operation; perhaps bit flip optimization
for hash tables should focus on these operations, something we plan to
investigate in the future.

\paragraph{Cache Effects}
The data structures we tested all had the same behavior---a warm-up period
where the cache system absorbed many of the writes followed by a period of
proportional increasing of bit flips as the number of update operations
increased. We must
keep this in-mind when evaluating data structures for bit flips, since we must
ensure that the ranges of inputs we test reach the expected scale for
our data structures, or we may be blind to its true behavior.
The cache size affects this, of course, but it does so in a predictable way in
the case of \xrbt, with only the warm-up period being extended by an amount
proportional to cache size. Of course, the behavior may be heavily dependent on write patterns.
Thus, we recommend further experiments and that system designers take caches into account when
evaluating bit flip behavior of their systems.

The cache additionally affects the read amplification seen in XOR linked lists,
wherein the XOR linked list implementation issues more reads than a
doubly-linked list implementation. However, the reads that make it to memory are
the same between the two, indicating that those extra reads are always in-cache.
The resultant write reduction and bit flip reduction is well worth the cost
since a read from cache is significantly cheaper than a write to memory.

\iffalse
    Data structures often have linear behavior for bit flips after a warmup.

    Number of bytes often not too different, due to caches

    cache sizes change things as you'd expect, cache gets overrun

    These data structures are new and not new. Optimizations that only make sense
    for BF. They're not theory advances, so haven't been looked at.

    We can worry about bit flips in software.

    Trading writes for reads.

    Prediction is possible.

    Absolute numbers!
    Malloc has relatively small impact

    Minimizing writes isn't enough.

    XOR with self address is interesting

    Register allocations, frame layouts, etc.






    Reiterate the lessons learned, highlighting what we need to keep in mind when
    designing data structures.

    Some of these data structures are kinda novel... talk about that? Also about how
    some of them are optimizations that you wouldn't ever do without \NVM because
    they're crazy, but with bit flips they become attractive.

    No one thought about this before because they're the SAME data structures and
    algorithms... no new theory advances, just practical ones.

    We can worry about bit flips in software, and be right. We can reduce them,
    relatively easily.

    Effects on security?

    trading writes for reads...

    Make sure to hit all the key points in the OUTLINE.

    Prediction is possible with instrumentation.

    Some things are surprising, or change at other scales (40 vs 48).

    Absolute numbers!

    etc.

\fi

