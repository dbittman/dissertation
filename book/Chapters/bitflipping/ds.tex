\section{Reducing Bit Flips in Software}
\label{sec:datastruct}

By reducing bit flips in software, we can effect improvements in \NVM lifetime
and power use without the need for hardware changes.
To build data structures to reduce bit
flips (Sections~\ref{sec:xorll_design}--\ref{sec:xrbt_design}), we propose
several optimizations to pointer storage along with additional optimizations for
indicating occupancy. For stack writes, we propose changes to compilers
to spill registers such that they avoid writing different registers to the same
place in the stack (Section~\ref{sec:stack}).

\subsection{XOR Linked Lists}
\label{sec:xorll_design}

XOR linked lists~\cite{xorll} are a memory-efficient doubly-linked list design
where, instead of storing a previous and next node pointer, each node stores
only a \texttt{siblings} value that is the XOR between the previous and next
node. If the previous node is at address $p$ and the next node is
at address $n$, the node stores $siblings = p \oplus n$. This scheme cuts the
number of stored pointers per node in half while still allowing bi-directional
traversal of the list---having pointers to two adjacent nodes is sufficient to
traverse both directions. However, an XOR linked list has
disadvantages; it does not allow $O(1)$ removal of a node with just a
single pointer to that node, as a node's siblings cannot be determined from the
node alone, and it increases code complexity by requiring XOR operations before
pointers are dereferenced.

When they were proposed, XOR linked lists had little advantage over doubly linked lists beyond
a modest memory saving.
However, with the need for fewer bit flips on \NVM, they gain a critical
advantage: they cut the number of stored pointers in half, reducing writes, but
they also store the XOR of two pointers, which are likely to contain similar
higher-order bits, making the \texttt{siblings} pointer mostly zeros.

One problem with the original design for XOR linked lists is that each node
stores $siblings = p \oplus n$, but for the first and last node, $p$ or $n$ are
\texttt{NULL}, so the full pointer value for its adjacent node is stored in the
head and tail. To further cut down on bit flips, we changed this design so that
the head and tail XOR their adjacent nodes with themselves (if the node at
address $h$ is the head, then it stores $siblings = h \oplus n$ instead of
$siblings = 0 \oplus n$). The optimization here is \textit{not} a performance
optimization---in fact, it's likely to reduce performance---and only makes sense
in the context of bit flips, an optimization goal that would not be
targeted before the introduction of \NVM. However, with bit flips in-mind, it
becomes critical. Other data structures may have similar optimizations that we
can easily make to reduce bit flips~\sidenote{Circular linked
	lists solve the head and tail siblings pointer problem automatically, since no pointers are
	stored as \texttt{NULL}; however, in XOR linked lists this increases the number of pointer
	updates during an insert operation and requires storing two adjacent head nodes to traverse.
}.

\subsection{XOR Hash Tables}
\label{sec:xht_design}

A direct application of XOR linked lists is chained hashing, a common
technique for dealing with hash table collisions~\cite{clrs}. An array of linked list heads
is maintained as the hash table, and when an item is inserted, it is appended to
the list at the bucket that the item hashes to. To optimize for bit flips,
we can store an XOR list instead of a normal linked list, but
since bidirectional traversal is not needed in a hash table
bucket, we need not complicate the implementation with a full XOR linked
list. Instead, we apply the property of XOR linked lists that we find
useful---XORing pointers.

Each pointer in each list node is XORed with the address of the node that
contains that pointer. For example, a list node $n$ whose next node is $p$ will
store $n \oplus p$ instead of $p$. In effect, this stores the distance between
the nodes rather than the absolute address of the next node and exploits
locality in memory allocators. The end of the list is marked with a \texttt{NULL}
pointer. In addition to a distance pointer, each node contains a key and a
pointer to a value. The list head stored in the hash table is a full node,
allowing access to the first entry in the list without needing to follow a
pointer.

A second optimization we make is that an empty list can be marked in one of two ways: the
least-significant bit (LSB) of the \texttt{next} pointer set to one, or the data pointer set to
\texttt{NULL}. When we initialize the table, it is set to zero everywhere, so the data pointers
are \texttt{NULL}.  During delete, if the list becomes empty, the LSB of the next
pointer in the list head is set to 1, a value it would never have when part of a list. This
allows the data pointer to remain set to a value such that when it is later overwritten, fewer bits
need to change. This is an example of an optimization that only makes sense in the context of bit
flips, as it increases code complexity for no other gain.

\subsection{XOR Red-Black Trees}
\label{sec:xrbt_design}

Binary search trees are commonly used for data indexing, support range queries,
and allow efficient lookup and modification, as long as they are balanced.
Red-black trees~\cite{rbt,clrs} are a common balanced binary tree data structure with
strictly-bounded rebalancing operations during modification. A typical red-black
tree (RBT) node contains pointers to its left child and right child, along with
meta-data. They often also contain a pointer to the parent node, since this
enables easier balancing implementation and more efficient range-query support
without significantly affecting performance due to the increased memory
usage~\cite{rbt_llh}.

We can generalize XOR linked lists to \textit{XOR trees}. Instead of storing
\texttt{left}, \texttt{right}, and \texttt{parent} pointers, each node stores
\texttt{xleft} and \texttt{xright}, which are the XOR between each child and
the parent addresses. This reduces the memory usage to the two-pointer case
while maintaining the benefits of having a parent pointer, since given a node
and one of its children (or its parent), we can traverse the entire tree. Like
XOR linked lists, the root node stores \texttt{xleft = root $\oplus$ left}, where
\texttt{root} is the address of the root node and \texttt{left} is the address
of its left child, saving bit flips. To indicate that a node has no left or right child, it
stores \texttt{NULL}.

Determining the child of a node requires both the node and its parent:\\
\texttt{
	get\_left\_child(Node *node, Node *parent) \{
	\indent return (parent $\oplus$ node->xleft); \\
	\}
}\\
Getting a node's parent, however, requires additional work. Given a child $c$ and a
node $n$, getting $n$'s parent requires we know \textit{which} child (left or
right) $c$ is. Fortunately, in a binary search tree we store the key $k$ of a node
in each node, and the nodes are well-ordered by their $k$. Thus, getting the
parent works as follows:\\
\texttt{
	get\_parent(Node *n, Node *c) \{\\
	\indent if(c->k < n->k) return (n->xleft $\oplus$ c);\\
	\indent else return (n->xright $\oplus$ c);\\
	\}}\\
Note that this is not the only way to disambiguate between pointers. In fact,
it's not strictly necessary to do so because the algorithms can be implemented
recursively without ever needing to traverse up the tree explicitly. However, providing upwards
traversal can reduce the complexity of implementation and improve the performance of iteration over
ranges. Another
solution to getting the parent node would be to record whether a node is a left
or right child by storing an extra bit along with the color. We did not
evaluate this method, as it would increase both writes and bit flips over our method.

\newcommand{\xrbt}{\texttt{xrbt}\xspace}
\newcommand{\xrbtbig}{\texttt{xrbt-big}\xspace}
\newcommand{\rbt}{\texttt{rbt}\xspace}

With these helper functions, we implemented both an XOR red black
tree (\xrbt) and a normal red-black tree (\rbt) using similar algorithms.
The code for \xrbt was just 20 lines longer, with only a
minor increase in code complexity. Node size was smaller in \xrbt,
with a node being 40 bytes instead of 48 bytes as in \rbt. To control for
the effects of node size on performance and bit flips, we built a variant of
\xrbt with the same code but with a node size of 48 bytes (\xrbtbig).


\paragraph{Generalization} These techniques can generalize beyond a red-black tree. Any ordered
$k$-ary tree can use XOR pointers in the same way. As discussed above, disambiguating between pointers
during traversal depends on either additional bits being stored or using an ordering property.
Either technique can work with arbitrary graph nodes.

\subsection{Stack Frames}
\label{sec:stack}

Data structure layout and data writes are only some of the writes made by a
program. Register spills, callee-saved register saving,
and return addresses pushed during function calls are all writes to memory, and
if these writes make it to \NVM, they will cause bit flips as well. These writes
may make it to main memory if the cache is saturated or if the program is
designed to keep program state in \NVM to enable instantaneous restart after
power cycles~\cite{narayanan:asplos12}. Additionally, systems designed for \NVM may run with write-through
caches to reduce consistency complexity, resulting in execution state reaching
\NVM.

The exact pattern of stack writes depends on the ABI and the calling convention of a system and
processor, though we focus on x86-64 Linux systems. When a program calls a function, it
(potentially) pushes a number of arguments to the stack, followed by a return address.  In the
called function, callee-saved registers are pushed to the stack, but \emph{only} if they are modified
during that function's execution. When finished, the callee pops all the saved registers and
returns.

Our observation is that the order that callee-saved registers are pushed to the
stack is \textit{not specified}, meaning that two different functions could push
the same registers in a different order. Secondly, the same callee-saved register
is less-likely to change drastically in a small amount of code in a tight loop,
since these registers are typically used for loop counters or bases for
addressing. Thus, a loop that calls two functions alternately will likely have
similar or the same values in the callee-saved registers during the invocation
of both functions.
If these two functions push the (often unchanged)
callee-saved registers to the same place both times, fewer bit
flips will occur than if the functions pushed them in different orders.
While this is just a simple example,
such loops that call out to alternating functions with different characteristics can occur, for
example, when rehashing a table, rebuilding a tree, or reading task items from a linked
list.

We propose specifying a callee-saved register frame layout that functions adhere
to, so that the registers are always pushed in the same order. To handle
variable numbers of arguments, we make use of passing arguments in registers,
common in many modern ABIs. If a function
need not push any callee-saved registers, it can still reserve the stack space
for that frame and then not push anything to save writes. Functions which only
save a small number of registers can still push them to the correct locations
within the frame. Finally, if this is standardized, programs need not worry
about library calls increasing bit flips.

\newcommand{\ffoo}{\texttt{A}\xspace}
\newcommand{\fbar}{\texttt{B}\xspace}
For example, if we have two functions \ffoo and \fbar in an ABI where registers
$e$, $f$, $g$, $h$ are callee-saved, and \ffoo uses $e$ while \fbar uses $g$, then
traditionally each function would simply push the frame pointer followed by the
register they wish to save:\\

\begin{minipage}{0.5\columnwidth}
	\texttt{A:\\
		\indent push fp\\
		\indent mov fp $\leftarrow$ sp\\
		\indent push e\\
		\indent ...\\
		\indent pop e\\
		\indent pop fp; ret
	}
\end{minipage}%
~
\begin{minipage}{0.5\columnwidth}
	\texttt{B:\\
		\indent push fp\\
		\indent mov fp $\leftarrow$ sp\\
		\indent push g\\
		\indent ...\\
		\indent pop g\\
		\indent pop fp; ret
	}
\end{minipage}\\

If $e$ and $g$ are significantly different, then a significant amount of needless
bit flips could occur if these functions are called often.
Instead, if we define a layout that functions adhere to for register saving, the
code would look like:\\

\begin{minipage}{0.5\columnwidth}
	\texttt{A:\\
		\indent push fp\\
		\indent mov fp $\leftarrow$ sp\\
		\indent push e\\
		\indent sub sp, 24\\
		\indent ...\\
		\indent add sp, 24\\
		\indent pop e\\
		\indent pop fp; ret
	}
\end{minipage}%
~
\begin{minipage}{0.5\columnwidth}
	\texttt{B:\\
		\indent push fp\\
		\indent mov fp $\leftarrow$ sp\\
		\indent sub sp, 16\\
		\indent push g\\
		\indent sub sp, 8\\
		\indent ...\\
		\indent add sp, 8\\
		\indent pop g\\
		\indent add sp, 16\\
		\indent pop fp; ret
	}
\end{minipage}\\

Here the code always pushes the same register to the same place, regardless of
the registers it needs to save, thereby allowing overwrites by likely similar values.
While it does add some additional instructions,
code could instead write registers directly to the stack locations using offset
style addressing, reducing code size.

