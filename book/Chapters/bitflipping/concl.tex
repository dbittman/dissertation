\section{Conclusion}
\label{sec:conclusion}

The pressures from new storage hardware trends compel us to explore new optimization goals as BNVM
becomes more common as a persistent store; the read/write asymmetry of BNVM must be addressed by
reducing bit flips.
%As BNVM becomes common as a persistent store, the pressures from new storage
%hardware trends compel us to explore new optimization goals---recognizing
%the read/write asymmetry of BNVM by reducing bit flips.
As we showed, the
number of raw writes is not always a faithful proxy for the number of bit flips,
so simple techniques that minimize writes overall may be ineffective.
At the OS level, we can reconsider memory allocator design to
minimize bit flips as pointers are written. Different data structures use and
write pointers in different ways, leading to different tradeoffs for data
structures when considering BNVM applications. At the compiler level, we show
that careful layout of stack frames can have a significant impact on bit flips
during program operation. Since it can be challenging to reason directly about
how application-level writes translate to raw writes due to the compiler and
caches, more sophisticated profiling tools are needed to help navigate the
tradeoffs between performance, consistency, power use, and wear-out.

Most importantly, we demonstrated the value of reasoning at the \emph{application level} about bit
flips, reducing bit flips by $1.13-3.56\times$ with minor code changes, no significant increase in
complexity, and little performance loss. The data structures we studied had novel implementations,
but were \textit{algorithmically} the same as their standard implementations; yet we still saw
dramatic improvements with little effort. This indicates that reasoning about bit flips in software
can yield significant improvements over in-hardware solutions and opens the door for additional
research at a variety of levels of the stack for bit flip reduction.  These techniques translate
directly to power saving and lifetime improvements, both important optimizations for early adoption
of new storage trends that will have lasting impact on systems, applications, and hardware.

%optimizations that are important for the early adoption of new storage trends and
%will have lasting impacts on systems, applications, and hardware.


%both system designers and hardware manufacturers.

%Perhaps surprisingly, the data
%structures that we studied were not novel!  Rather, each was either a
%``classic’’ but impractical structure (e.g. the XOR linked list) or a
%straightforward adaptation of an existing structure (e.g. the XOR hash table).
%The extreme asymmetry in BNVM power consumption merits the reconsideration of
%our choice of data structures, in some cases favoring those specifically
%designed to minimize bit flips.  We are excited to re-explore ingenious but
%heretofore impractical data structures and algorithms under the pressure of
%these hardware trends. 



\iffalse
\begin{prior}


Minimizing bit flips is an important and challenging design goal for BNVM-aware
data structures. Although minimizing writes \textit{can} be a
good proxy for minimizing bit flips, it is not always one; thus, we must be sure
to incorporate bit flips as an explicit measurement when evaluating a data
structure's performance and fit for BNVM. Even in the case where both writes and
bit flips are reduced, simple design choices can allow the bit flip reduction to
vastly exceed the expected reduction from simply reducing writes,
indicating that we can improve over blindly reducing writes by either choosing
which writes to minimize and when or by making those writes cheaper. Write
reduction techniques may not reduce bit flips equally, so profiling bit flips
directly is the only viable mechanism to determine the behavior of a
data structure.  Finally, we found that a few straight-forward implementation tweaks to a
simple data structure resulted in significant savings, and many data
structures can be similarly optimized.

Although we do not expect every data structure to be easily optimized for bit
flips, and although we acknowledge that optimizing for bit flips can be
difficult without knowing something about the workload, how it changes, and how
data is distributed,
there is research that can be done towards common and
often applicable techniques, ranging from which data structures and algorithms
to choose, to how to best encode data structures in memory to reduce bit flips,
and to how these techniques will apply or change with intermediate hardware
layers such as caching, write ordering, and wear leveling.
We plan to further this work by doing these experiments on real
hardware and full system simulators.
These questions will require more research to begin to answer, but
we believe they are necessary to answer as BNVM grows in
popularity.

\end{prior}

\fi

