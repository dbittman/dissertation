\chapter{Far Out Memory Hierarchies}\label{ch:farout}

\section{Persistence and Distibution}


\unedit{
    The first key characteristic of \NVM is
    low latency: only $1.5$--$8\times$ the
    latency of DRAM in most cases~\cite{ucsd_bnvm}. Thus the cost of a system call to
    access \NVM
    %($0.5$--\SI{1}{\micro\second})
    dominates the
    latency of the access itself. The second key characteristic is that the processor can directly
    access persistent storage using load and store instructions.
    Direct, low latency access to \NVM means that explicit
    serialization is a poor fit---it adds complexity, as programmers must maintain different
    data formats and the transformations between them, and the overhead is intolerable due to \NVM's
    low latency. Hence, we should design the semantics of the programming model around
    \emph{in-memory} persistent data structures, giving programs direct access to them without
    explicit persistence calls or serialization methods.
}


\section{Memory Capacity}

\section{Interconnect Technologies}

\section{Why Now?}
